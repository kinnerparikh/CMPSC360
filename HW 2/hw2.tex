\documentclass{article} % This command is used to set the type of document you are working on such as an article, book, or presenation

\usepackage{geometry} % This package allows the editing of the page layout
\usepackage{amsmath}  % This package allows the use of a large range of mathematical formula, commands, and symbols
\usepackage{graphicx}  % This package allows the importing of images
\usepackage{soul}
\usepackage{amsfonts}
\usepackage{dirtytalk}

\newcommand{\question}[2][]{\begin{flushleft}
        \textbf{Question #1}: \textit{#2}

\end{flushleft}}
\newcommand{\sol}{\textbf{Solution}:} %Use if you want a boldface solution line
\newcommand\tab[1][0.4cm]{\hspace*{#1}}
\newcommand{\maketitletwo}[2][]{\begin{center}
        \Large{\textbf{Homework #1}
            
            CMPSC 360} % Name of course here
        \vspace{5pt}
        
        \normalsize{Kinner Parikh  % Your name here
        
        \today}        % Change to due date if preferred
        \vspace{15pt}
        
\end{center}}
\begin{document}
    \maketitletwo[2]  % Optional argument is assignment number
    %Keep a blank space between maketitletwo and \question[1]
    
    \question[1]{}
    
    1. This is a statement because the truth value of the statement can be determined: Obama was the president during 2010 or not

    2. This is a statement becuase the quantity x + 3 could be a positive integer or not, which means the truth value can be determined

    3. This is a statement because 15 is either an odd number or it is not

    4. 

    \question[2]{}

    a) If 1 + 1 = 3 $\rightarrow$ F, then dogs can fly $\rightarrow$ F $\Rightarrow$  \hl{T}

    b) If 1 + 1 = 2 $\rightarrow$ T, then dogs can fly $\rightarrow$ F $\Rightarrow$  \hl{F}

    c) If 2 + 2 = 4 $\rightarrow$ T, then 1 + 2 = 3 $\rightarrow$ T $\Rightarrow$  \hl{T}

    \question[3]{}

    1. a

\end{document}