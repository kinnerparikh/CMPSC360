\documentclass{article} % This command is used to set the type of document you are working on such as an article, book, or presenation

\usepackage{geometry} % This package allows the editing of the page layout
\usepackage{amsmath}  % This package allows the use of a large range of mathematical formula, commands, and symbols
\usepackage{graphicx}  % This package allows the importing of images
\usepackage{soul}
\usepackage{amsfonts}
\usepackage{dirtytalk}
\usepackage{tabto}
\usepackage{xcolor,colortbl, amssymb}

% https://www.messletters.com/en/big-text/

\newcommand{\question}[2][]{\begin{flushleft}
        \textbf{Question #1}: \textit{#2}

\end{flushleft}}

\definecolor{Green}{rgb}{0, 1, 0}
\definecolor{Pink}{rgb}{1, .753, .796}

\newcommand{\sol}{\textbf{Solution}:} %Use if you want a boldface solution line
%\newcommand\tab[1][0.4cm]{\hspace*{#1}}
\newcommand{\maketitletwo}[2][]{\begin{center}
        \Large{\textbf{Quiz #1}
            
            CMPSC 360} % Name of course here
        \vspace{5pt}
        
        \normalsize{Kinner Parikh  % Your name here
        
        \today}        % Change to due date if preferred
        \vspace{15pt}
        
\end{center}}
\begin{document}
    \maketitletwo[6]  % Optional argument is assignment number
    %Keep a blank space between maketitletwo and \question[1]
    
    \question[1]{}

    $a_n = (n + 1)! - 1$ for $n \geq 1$

    \question[3]{}

    \begin{alignat*}{3}
        & f(x)        &&=\ \frac{4x+3}{2x+5} \\
        & y           &&=\ \frac{4x+3}{2x+5} \ \text{(by definition of f)} \\
        & 2xy +5y     &&=\ 4x + 3 \\
        & 5y - 3      &&=\ 4x-2xy \\
        & 5y - 3      &&=\ x(4-2y) \\
        & \frac{5y - 3}{4-2y} &&=\ x \\
        & f'(x) &&=\ \frac{5x - 3}{4-2x}
    \end{alignat*}

    \question[4]{}

    There are 17 total children and there are 7 days in a week. Considering there are more children than there are days in a week, and since $\lfloor 17/7 \rfloor = 2$, we can say that at least three of them were born on the same day of the week.
    
\end{document}