\documentclass{article} % This command is used to set the type of document you are working on such as an article, book, or presenation

\usepackage{geometry} % This package allows the editing of the page layout
\usepackage{amsmath}  % This package allows the use of a large range of mathematical formula, commands, and symbols
\usepackage{graphicx}  % This package allows the importing of images
\usepackage{soul}
\usepackage{amsfonts}
\usepackage{dirtytalk}
\usepackage{tabto}
\usepackage{xcolor,colortbl, amssymb}

\newcommand{\question}[2][]{\begin{flushleft}
        \textbf{Question #1}: \textit{#2}

\end{flushleft}}
\newcommand{\sol}{\textbf{Solution}:} %Use if you want a boldface solution line
\definecolor{Green}{rgb}{0, 1, 0}
\newcommand{\maketitletwo}[2][]{\begin{center}
        \Large{\textbf{Quiz #1}
            
            CMPSC 360} % Name of course here
        \vspace{5pt}
        
        \normalsize{Kinner Parikh  % Your name here
        
        \today}        % Change to due date if preferred
        \vspace{15pt}
        
\end{center}}
\begin{document}
    \maketitletwo[2]  % Optional argument is assignment number
    %Keep a blank space between maketitletwo and \question[1]
    
    \question[1]{}

    $\neg r \lor (p \land q)$

    $(\neg r \lor p) \land (\neg r \lor q)$ Distributive Rules

    $(r \rightarrow p) \land (r \rightarrow q)$ Conditional Equivalence

    \begin{table}[h]
        \centering
        \caption{$\neg r \lor (p \land q) \equiv (r \rightarrow p) \land (r \rightarrow q)$ }
        \begin{tabular}{c | c | c | c | c | c | c |>{\columncolor{Green}} c |>{\columncolor{Green}} c}
            $p$ & $q$ & $r$ & $ \neg r$ & $p \land q$ & $r \rightarrow q$ & $r \rightarrow p$ & $\neg r \lor (p \land q)$ & $(r \rightarrow p) \land (r \rightarrow q)$\\
            \hline
            T & T & T & F & T & T & T & T & T \\
            T & T & F & T & T & T & T & T & T \\
            T & F & T & F & F & F & T & F & F \\
            F & T & T & F & F & T & F & F & F \\
            T & F & F & T & F & T & T & T & T \\
            F & T & F & T & F & T & T & T & T \\
            F & F & T & F & F & F & F & F & F \\
            F & F & F & T & F & T & T & T & T \\ 
            
        \end{tabular}
    \end{table}
    
\end{document}