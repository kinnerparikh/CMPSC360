\documentclass{article} % This command is used to set the type of document you are working on such as an article, book, or presenation

\usepackage{geometry} % This package allows the editing of the page layout
\usepackage{amsmath}  % This package allows the use of a large range of mathematical formula, commands, and symbols
\usepackage{graphicx}  % This package allows the importing of images
\usepackage{soul}
\usepackage{amsfonts}
\usepackage{dirtytalk}
\usepackage{tabto}
\usepackage{xcolor,colortbl, amssymb}

% https://www.messletters.com/en/big-text/

\newcommand{\question}[2][]{\begin{flushleft}
        \textbf{Question #1}: #2

\end{flushleft}}

\definecolor{Green}{rgb}{0, 1, 0}
\definecolor{Pink}{rgb}{1, .753, .796}

\newcommand{\sol}{\textbf{Solution}:} %Use if you want a boldface solution line
%\newcommand\tab[1][0.4cm]{\hspace*{#1}}
\newcommand{\maketitletwo}[2][]{\begin{center}
        \Large{\textbf{Homework #1}
            
            CMPSC 360} % Name of course here
        \vspace{5pt}
        
        \normalsize{Kinner Parikh  % Your name here
        
        \today}        % Change to due date if preferred
        \vspace{15pt}
        
\end{center}}
\begin{document}
    \maketitletwo[9]  % Optional argument is assignment number
    %Keep a blank space between maketitletwo and \question[1]
    
    \question[1]{Show that if $x$ is an odd integer, then $x^2$ has the form $8k + 1$, for some $k \in \mathbb{Z}$}

    \question[2]{Solve for $23 ^ 3 (\text{mod } 30)$}

    \question[3]{Show that if an integer $n$ is not divisible by 3, then $n^2 - 1$ is always divisible by 3. 
    Similarly, show that if an integer $n$ is not divisible by 3, then $n^3 - 1 \equiv 0$}

    \question[4]{Find GCD of 2947 and 3997 using Euclidean Theorem.}
    \begin{alignat*}{5}
        &3997\ &=\ &2947(1) &&+ 1050\\
        &2947  &=\ &1050(2) &&+ 847\\
        &1050  &=\ &847(1)  &&+ 203\\
        &847   &=\ &203(4)  &&+ 35\\
        &203   &=\ &35(5)   &&+ 28\\
        &35    &=\ &28(1)   &&+ 7\\
        &28    &=\ &7(4)    &&+ 0
    \end{alignat*}

    So, $gcd(2947, 3997) = 7$

    \question[5]{Express $gcd(128469, 12818)$ as a linear combination of 128469 and 12818 using 
    extended Euclid algorithm.}

    Applying Euclid's algorithm:
    \begin{alignat*}{5}
        &128469\ &=\ &12818(10) &&+ 289\\
        &12818   &=\ &289  (44) &&+ 102\\
        &289     &=\ &102  (2) &&+   85\\
        &102     &=\ &85   (1) &&+   17\\
        &85      &=\ &17   (5) &&+    0
    \end{alignat*}

    \begin{tabular}{|c|c|c|c|c|c|c|}
        \hline
        $i$ & $r_i$ & $r_{i + 1}$ & $q_{i + 1}$ & $r_{i + 2}$ & $s_i$ & $t_i$\\\hline
        0 & 128469 & 12818 & 10 & 289 & 1 & 0\\
        1 & 12818  & 289   & 44 & 102 & 0 & 1\\
        2 & 289    & 102   & 2  &  85 & 1 & -10\\
        3 & 102    & 85    & 1  &  17 & -44 & 441\\
        4 & 85     & 17    & 5  &   0 & 90 & -891\\
        5 &        &       &    &     & -224 & 2223 \\\hline 
    \end{tabular} 
    -224 * 128469 + 2223 * 12818

    \question[6]{Prove that if $a \mid bc$ with $gcd(a, b) = 1$, then $a \mid c$}

    \question[7]{Prove that $gcd(a^2,b^2)=gcd(a,b)^2$ using Bezout's identity.}

    \question[8]{For $Z_{11}$, find out:}
    
    a) $3 \oplus 7$

    b) $3 \otimes 7$

    c) $10 \ominus 7$

    d) $10 \oslash 7$

    \question[9]{Determine whether every element $a$ of $Z_n$ has an inverse for $n = 5, 6$ and $ 7, 11$}

    \question[10]{Write the following decimal string $334_{10}$ to senary (base 6) showing work}
    \begin{alignat*}{2}
        334 \div 6 &= 55\ &&R\ 4\\
        55 \div 6 &= 9 &&R\ 1\\
        9 \div 6 &= 1 &&R\ 3\\
        1 \div 6 &= 0 &&R\ 1
    \end{alignat*}

    So $334_{10} = 1314_6$.


\end{document}