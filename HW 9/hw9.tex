\documentclass{article} % This command is used to set the type of document you are working on such as an article, book, or presenation

\usepackage{geometry} % This package allows the editing of the page layout
\usepackage{amsmath}  % This package allows the use of a large range of mathematical formula, commands, and symbols
\usepackage{graphicx}  % This package allows the importing of images
\usepackage{soul}
\usepackage{amsfonts}
\usepackage{dirtytalk}
\usepackage{tabto}
\usepackage{xcolor,colortbl, amssymb}

% https://www.messletters.com/en/big-text/

\newcommand{\question}[2][]{\begin{flushleft}
        \textbf{Question #1}: #2

\end{flushleft}}

\definecolor{Green}{rgb}{0, 1, 0}
\definecolor{Pink}{rgb}{1, .753, .796}

\newcommand{\sol}{\textbf{Solution}:} %Use if you want a boldface solution line
%\newcommand\tab[1][0.4cm]{\hspace*{#1}}
\newcommand{\maketitletwo}[2][]{\begin{center}
        \Large{\textbf{Homework #1}
            
            CMPSC 360} % Name of course here
        \vspace{5pt}
        
        \normalsize{Kinner Parikh  % Your name here
        
        \today}        % Change to due date if preferred
        \vspace{15pt}
        
\end{center}}
\begin{document}
    \maketitletwo[9]  % Optional argument is assignment number
    %Keep a blank space between maketitletwo and \question[1]
    
    \question[1]{Show that if $x$ is an odd integer, then $x^2$ has the form $8k + 1$, for some $k \in \mathbb{Z}$}

    Proof:

    Assume that $x$ is an odd integer.

    By definition of odd, $x = 2t + 1$ where $t \in \mathbb{Z}$.

    This also means that $x = 4t + 1$ and $x = 4t + 3$.

    \textbf{Case 1:} $(x = 4t + 1)$
    
    \tabto{1cm} $x^2 = (4t + 1)^2$
         
    \tabto{1.45cm} $= 16t^2 + 8t + 1$

    \tabto{1.45cm} $= 8(2t^2 + t) + 1$

    \tabto{1.45cm} $= 8k + 1$ such that $k \in \mathbb{Z}$ where $k = 2t^2 + t$

    \tabto{1cm} So we know  when $x = 4t + 1$, that $x^2$ has the form $8k + 1$

    \textbf{Case 2:} $(x = 4t + 3)$

    \tabto{1cm} $x^2 = (4t + 3)^2$
         
    \tabto{1.45cm} $= 16t^2 + 24t + 9$
    
    \tabto{1.45cm} $= 16t^2 + 24t + 8 + 1$

    \tabto{1.45cm} $= 8(2t^2 + 3t + 1) + 1$

    \tabto{1.45cm} $= 8k + 1$ such that $k \in \mathbb{Z}$ where $k = 2t^2 + 3t + 1$
    
    \tabto{1cm} So we know when $x = 4t + 3$, that $x^2$ has the form $8k + 1$

    Both cases hold true. Therefore, when $x$ is an odd integer, then $x^2$ has the form $8k + 1$. $\square$

    \question[2]{Solve for $23 ^ 3 (\text{mod } 30)$}

    $23 ^ 3 (\text{mod } 30) = 23^{1 + 2} (\text{mod } 30)$

    \tabto{2.55cm} $= ((23^1 \text{ mod } 30)(23^2 \text{ mod } 30))\text{ mod } 30$

    \tabto{2.55cm} $= (23 \cdot 19) \text{ mod } 30$

    \tabto{2.55cm} $= 437 \text{ mod } 30$

    \tabto{2.55cm} $= 17$

    \newpage

    \question[3]{Show that if an integer $n$ is not divisible by 3, then $n^2 - 1$ is always divisible by 3. 
    Similarly, show that if an integer $n$ is not divisible by 3, then $n^2 - 1 \equiv 0$}

    Proof:

    Assume that $3 \nmid n$ such that $n \in \mathbb{Z}$

    We need to prove that $3 \mid (n^2 - 1)$ and $n^2 - 1 \equiv 0$, which means $n^2 = 1 (\text{mod } 3)$

    This means that $n = 3x + 1$ or $n = 3x + 2$ such that $x \in \mathbb{Z}$

    \textbf{Case 1:} $(n = 3x + 1)$

    \tabto{1cm} $n^2 - 1 = (3x + 1)^2 - 1$

    \tabto{2.07cm} $= 9x^2 + 6x + 1 - 1$

    \tabto{2.07cm} $= 9x^2 + 6x$

    \tabto{2.07cm} $= 3(3x^2 + 2x)$

    \tabto{2.07cm} $= 3t$ where $t \in \mathbb{Z}$ and $t = 3x^2 + 2x$

    \tabto{1cm} By definition of divides, when $n = 3x + 1$, then $3 \mid (n^2 + 1)$.

    \textbf{Case 2:} $(n = 3x + 2)$

    \tabto{1cm} $n^2 - 1 = (3x + 2)^2 - 1$

    \tabto{2.07cm} $= 9x^2 + 12x + 4 - 1$

    \tabto{2.07cm} $= 9x^2 + 12x + 3$

    \tabto{2.07cm} $= 3(3x^2 + 4x + 1)$

    \tabto{2.07cm} $= 3t$ where $t \in \mathbb{Z}$ and $t = 3x^2 + 4x + 1$

    \tabto{1cm} By definition of divides, when $n = 3x + 2$, then $3 \mid (n^2 + 1)$.

    Since both cases hold, we know that if $3 \nmid n$ such that $n \in \mathbb{Z}$, then $3 \mid (n^2 - 1)$. $\square$

    \question[4]{Find GCD of 2947 and 3997 using Euclidean Theorem.}
    \begin{alignat*}{5}
        &3997\ &=\ &2947(1) &&+ 1050\\
        &2947  &=\ &1050(2) &&+ 847\\
        &1050  &=\ &847(1)  &&+ 203\\
        &847   &=\ &203(4)  &&+ 35\\
        &203   &=\ &35(5)   &&+ 28\\
        &35    &=\ &28(1)   &&+ 7\\
        &28    &=\ &7(4)    &&+ 0
    \end{alignat*}

    So, $gcd(2947, 3997) = 7$

    \question[5]{Express $gcd(128469, 12818)$ as a linear combination of 128469 and 12818 using 
    extended Euclid algorithm.}

    Applying Euclid's algorithm:
    \begin{tabular}{|c|c|c|c|c|c|c|}
        \hline
        $i$ & $r_i$ & $r_{i + 1}$ & $q_{i + 1}$ & $r_{i + 2}$ & $s_i$ & $t_i$\\\hline
        0 & 128469 & 12818 & 10 & 289 & 1    & 0    \\
        1 & 12818  & 289   & 44 & 102 & 0    & 1    \\
        2 & 289    & 102   & 2  &  85 & 1    & -10  \\
        3 & 102    & 85    & 1  &  17 & -44  & 441  \\
        4 & 85     & 17    & 5  &   0 & 89   & -892 \\
        5 &        &       &    &     & -133 & 1333 \\\hline 
    \end{tabular}
    \vspace*{0.1cm}

    $gcd(128469, 12818) = (-133)(128469) + (1333)(12818)$

    \question[6]{Prove that if $a \mid bc$ with $gcd(a, b) = 1$, then $a \mid c$}

    Proof:

    Assume that $a \mid bc$ and $gcd(a, b) = 1$

    By definition of divides, we know that $bc = at$ where $t \in \mathbb{Z}$

    Dividing both sides by $b$, we get $c = \frac{at}{b}$

    Since we know that $gcd(a, b) = 1$, it must mean that $t \mid b$.

    Therefore, we can say that $\frac{t}{b} \in \mathbb{Z}$

    This means that $c = ax$ where $x = \frac{t}{b}$ 

    Thus, by the definition of divides, we can say that $a \mid c$. $\square$

    \question[7]{Prove that $gcd(a^2,b^2)=gcd(a,b)^2$ using Bezout's identity.}

    \question[8]{For $Z_{11}$, find out:}
    
    a) $3 \oplus 7$

    \tabto{0.9cm}$(3 + 7)\text{ mod } 11 = 10 \text{ mod } 11 = 10$

    b) $3 \otimes 7$

    \tabto{0.9cm}$(3 \cdot 7)\text{ mod } 11 = 21 \text{ mod } 11 = 10$

    c) $10 \ominus 7$

    \tabto{0.9cm}$(10 - 7)\text{ mod } 11 = 3 \text{ mod } 11 = 3$

    d) $10 \oslash 7$
    
    \tabto{0.95cm}$10 \otimes 7^{-1}= 10 \otimes 4 = (10 \cdot 4)\text{ mod } 11 = 40 \text{ mod } 11 = 7$

    \question[9]{Determine whether every element $a$ of $Z_n$ has an inverse for $n = 5, 6$ and $ 7, 11$}

    \question[10]{Write the following decimal string $334_{10}$ to senary (base 6) showing work}
    \begin{alignat*}{2}
        334 \div 6 &= 55\ &&R\ 4\\
        55  \div 6 &= 9   &&R\ 1\\
        9   \div 6 &= 1   &&R\ 3\\
        1   \div 6 &= 0   &&R\ 1
    \end{alignat*}

    So $334_{10} = 1314_6$.

    \question[11]{Find the GCD of 846 and 265.}

    1) 846 = $(q_1 = 3) * 265 + (r_1 = 51)$

    2) 265 = $(q_2 = 5) * r_1 + (r_2 = 10)$

    3) $r_1 = (q_3 = 5) * r_2 + (r_3 = 1)$

    4) $r_2 = (q_4 = 10) * r_3 + 0$

    $r_3$ is our gcd.

    
\end{document}