\documentclass{article} % This command is used to set the type of document you are working on such as an article, book, or presenation

\usepackage{geometry} % This package allows the editing of the page layout
\usepackage{amsmath}  % This package allows the use of a large range of mathematical formula, commands, and symbols
\usepackage{graphicx}  % This package allows the importing of images
\usepackage{soul}
\usepackage{amsfonts}
\usepackage{dirtytalk}
\usepackage{tabto}
\usepackage{xcolor,colortbl, amssymb}


\newcommand{\question}[2][]{\begin{flushleft}
        \textbf{Question #1}: \textit{#2}

\end{flushleft}}

\definecolor{Green}{rgb}{0, 1, 0}
\definecolor{Pink}{rgb}{1, .753, .796}

\newcommand{\sol}{\textbf{Solution}:} %Use if you want a boldface solution line
%\newcommand\tab[1][0.4cm]{\hspace*{#1}}
\newcommand{\maketitletwo}[2][]{\begin{center}
        \Large{\textbf{Homework #1}
            
            CMPSC 360} % Name of course here
        \vspace{5pt}
        
        \normalsize{Kinner Parikh  % Your name here
        
        \today}        % Change to due date if preferred
        \vspace{15pt}
        
\end{center}}
\begin{document}
    \maketitletwo[1]  % Optional argument is assignment number
    %Keep a blank space between maketitletwo and \question[1]
    
    \question[1]{}

    $[Y \land (\neg X  \land \neg Y)] \lor [X \land [(X \land Y) \lor (\neg X \land Y) \lor (X \land \neg Y)]]$

    $[(Y \land \neg Y) \land \neg X] \lor [X \land [(X \land Y) \lor (\neg X \land Y) \lor (X \land \neg Y)]]$ Associative Rule

    $[F \land \neg X] \lor [X \land [(X \land Y) \lor (\neg X \land Y) \lor (X \land \neg Y)]]$ Contradiction

    $[F \land \neg X] \lor [X \land [(X \land Y) \lor [(\neg X \lor X) \land (Y \lor \neg Y)]]]$ Distributive Law

    $[F \land \neg X] \lor [X \land [(X \land Y) \lor [T \land T]]]$ Tautology

    $[F \land \neg X] \lor [X \land [(X \land Y) \lor T]]$ Identity

    $F \lor [X \land [(X \land Y) \lor T]]$ Identity

    $F \lor [X \land (X \land Y)]$ Identity

    $F \lor [(X \land X) \land Y]$ Associative Rule

    $F \lor [X \land Y]$ Idempotence

    $X \land Y$ Identity



    
\end{document}