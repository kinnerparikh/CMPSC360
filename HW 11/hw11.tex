\documentclass{article} % This command is used to set the type of document you are working on such as an article, book, or presenation

\usepackage{geometry} % This package allows the editing of the page layout
\usepackage{amsmath}  % This package allows the use of a large range of mathematical formula, commands, and symbols
\usepackage{graphicx}  % This package allows the importing of images
\usepackage{soul}
\usepackage{amsfonts}
\usepackage{dirtytalk}
\usepackage{tabto}
\usepackage{xcolor,colortbl, amssymb}

% https://www.messletters.com/en/big-text/

\newcommand{\question}[2][]{\begin{flushleft}
        \textbf{Question #1}: \textit{#2}

\end{flushleft}}

\newcommand{\comb}[2][]{
    {}_{#1} C_{#2}
}

\newcommand{\perm}[2]{
    {}_{#1} P_{#2}
}

\newcommand{\fl}[1]{
    \lfloor \frac{10000}{#1} \rfloor
}

\definecolor{Green}{rgb}{0, 1, 0}
\definecolor{Pink}{rgb}{1, .753, .796}

\newcommand{\sol}{\textbf{Solution}:} %Use if you want a boldface solution line
%\newcommand\tab[1][0.4cm]{\hspace*{#1}}
\newcommand{\maketitletwo}[2][]{\begin{center}
        \Large{\textbf{Homework #1}
            
            CMPSC 360} % Name of course here
        \vspace{5pt}
        
        \normalsize{Kinner Parikh  % Your name here
        
        \today}        % Change to due date if preferred
        \vspace{15pt}
        
\end{center}}
\begin{document}
    \maketitletwo[11]  % Optional argument is assignment number
    %Keep a blank space between maketitletwo and \question[1]
    
    \question[1]{}

    a) How many edges are there in a graph with 10 vertices, each having a degree 3? | \textbf{15}

    b) How many edges are there in a graph with 8 vertices, having a degree 1,1,2,2,3,3,3,3 
    \tabto{1cm}respectively? | \textbf{9}

    c)  How many vertices are there in a graph with 19 edges, having 3 vertices of degree 4 and all 
    \tabto{0.96cm}the other vertices are of degree 2? | \textbf{13}

    \question[2]{}

    With repetition: $6! = 720$

    Without repetition: $\frac{6!}{2} = 360$

    \question[3]{}

    a) $\frac{10!}{2!} = 1814400$

    b) $\frac{8!}{2!} \cdot 5! = 2419200$

    c) $\frac{10!}{2!} \cdot 2 \cdot 7 = 10! \cdot 7 = 25401600$

    \question[4]{}

    ${}_2 C_1 \cdot {}_5 C_2 + {}_2 C_2 \cdot {}_5 C_1 = 25$

    \question[5]{}

    $4^3 - 1 = 63$

    \question[6]{}

    No

    \question[7]{}

    \newpage

    \question[8]{}

    The five cases are: $x_2 = \{0, 1, 2, 3, 4\}$

    When $x_2 = 0$

    \tabto{1cm} $x_1 + x_3 + x_4 = 10$, so ${}_{12} C_2$

    When $x_2 = 1$

    \tabto{1cm} $x_1 + x_3 + x_4 = 9$, so ${}_{11} C_2$

    When $x_2 = 2$

    \tabto{1cm} $x_1 + x_3 + x_4 = 8$, so ${}_{10} C_2$

    When $x_2 = 3$

    \tabto{1cm} $x_1 + x_3 + x_4 = 7$, so ${}_{9} C_2$

    When $x_2 = 4$

    \tabto{1cm} $x_1 + x_3 + x_4 = 6$, so ${}_{8} C_2$

    So, ${}_{12} C_2 + {}_{11} C_2 + {}_{10} C_2 + {}_{9} C_2 + {}_{8} C_2 = 230$

    \question[9]{}

    $\comb[4]{0} \cdot 3^0 \cdot (2x)^4 + \comb[4]{1} \cdot 3^1 \cdot (2x)^3 + \comb[4]{2} \cdot 3^2 \cdot (2x)^2 + \comb[4]{3} \cdot 3^3 \cdot (2x)^1 + \comb[4]{4} \cdot 3^4 \cdot (2x)^0$

    $= 16x^4 + 96x^3 + 216x^2 + 216x + 81$

    \question[14]{}

    $\fl{3} + \fl{5} + \fl{7} + \fl{11} - \fl{15} - \fl{21} - \fl{33} - \fl{35} - \fl{55} - \fl{33}$

    \vspace{0.2cm}
    $- \fl{105} - \fl{165} - \fl{231} - \fl{385} - \fl{1155} = 5485$

    \question[15]{}

    $\perm{20}{15} = \frac{20!}{15!} = 1860480$

    \question[16]{}

    $\perm{6}{3} + \perm{6}{2} = 120 + 30 = 150$

    $\perm{7}{2} + 6 + 6 = 42 + 12 = 54$

    So in total, they can watch 150 + 54 = \underline{204 movies}


    
\end{document}