\documentclass{article} % This command is used to set the type of document you are working on such as an article, book, or presenation

\usepackage{geometry} % This package allows the editing of the page layout
\usepackage{amsmath}  % This package allows the use of a large range of mathematical formula, commands, and symbols
\usepackage{graphicx}  % This package allows the importing of images
\usepackage{soul}

\newcommand{\question}[2][]{\begin{flushleft}
        \textbf{Question #1}: \textit{#2}

\end{flushleft}}
\newcommand{\sol}{\textbf{Solution}:} %Use if you want a boldface solution line
\newcommand\tab[1][0.4cm]{\hspace*{#1}}
\newcommand{\maketitletwo}[2][]{\begin{center}
        \Large{\textbf{Assignment #1}
            
            CMPSC 360} % Name of course here
        \vspace{5pt}
        
        \normalsize{Kinner Parikh  % Your name here
        
        \today}        % Change to due date if preferred
        \vspace{15pt}
        
\end{center}}
\begin{document}
    \maketitletwo[1]  % Optional argument is assignment number
    %Keep a blank space between maketitletwo and \question[1]
    
    \question[1]{} 
    
    Since there are 8 such prime numbers that are less than 20 (2, 3, 5, 7, 11, 13, 17, 19) the power set of S has a cardinality of $2^8$, or 256 subsets
    
    \question[2]{}
    
    a. $\{ a, b, \{c, d\}, e, f, g, h \}$

    b. $\{ a, b, \{c, d\}, e \}$

    c. $\emptyset$
    
    d. $\{ f, g, h \}$
    
    
    \question[3]{}

    a) $\overline{A} \cup B$ by DeMorgan's Law

    \tab therefore, $\overline{A} = \{0, 1, 4, 5, 8, 9, 12, 15 \}$, B = $\{ 1, 4, 5, 8, 9 \}$

    \tab so $\{0, 1, 4, 5, 8, 9, 12, 15 \} \cup \{ 1, 4, 5, 8, 9 \}$

    \tab = \hl{$\{ 0, 1, 4, 5, 8, 9, 12, 15 \}$}

    b) $A\times(U - A - B)$

    \tab so $U - A = \{0, 1, 4, 5, 8, 9, 12, 15 \}$

    \tab therefore, $U - A - B = \{ 0, 12, 15 \}$

    \tab so, $A \times \{ 0, 12, 15 \}$ 

    \tab = \hl{$\{(2,0), (2,12), (2,15), (3,0), (3,12), (3,15), (6,0), (6,12), 
    (6,15), (7,0), (7,12), (7,15), (10,0), (10,12), (10,15)\}$}

    \question[4]{}

    a) $\emptyset$

    b) $\{ 9 \}$

    \question[5]{}

    a) $\emptyset$

    b) $C \cup B = \{ 0, 1, \alpha, \beta \}$

    \tab $\{ 0, 1, \alpha, \beta \} \cup A = \{0, 1, \alpha, \beta, X, Y, Z \}$
    
    \tab $B \times \{0, 1, \alpha, \beta, X, Y, Z \} = \{0, 1\} \times \{0, 1, \alpha, \beta, X, Y, Z \}$
        
    \tab = \hl{$\{ (0, 0), (0, 1), (0, \alpha), (0, \beta), (0, X), (0, Y), (0, Z), (1, 0), (1, 1), (1, \alpha), (1, \beta), (1, X), (1, Y), (1, Z)\}$}

    \question[6]{}

    a) 19

    b) 8

    \question[7]{}

    $A \times C = \{ \alpha, \beta, \gamma \} \times \{6, 8\} = \{ (\alpha, 6), (\alpha, 8), (\beta, 6), (\beta, 8), (\gamma, 6), (\gamma, 8) \} $ 

    $B \times C = \{ (c, 6), (c, 8), (d, 6), (d, 8), (\gamma, 6), (\gamma, 8) \}$

    $\{ (\alpha, 6), (\alpha, 8), (\beta, 6), (\beta, 8), (\gamma, 6), (\gamma, 8) \} \times \{ (c, 6), (c, 8), (d, 6), (d, 8), (\gamma, 6), (\gamma, 8) \}$

    therefore $ ( A \times C ) \cap ( B \times C ) =$
    \hl{$\{ (\gamma, 6), (\gamma, 8) \}$}

    \question[8]{}

    a) This is a proposition, True
    
    b) This is a proposition, False
    
    c) This is a proposition, True

    \question[9]{Collaboration with Sharon Liu and Sahil Kuwadia}

    Show that $\overline{\overline{A} \cap B} = A \cup \overline{B}$

    Suppose x $\in{ \overline{\overline{A} \cap B}} $

    By definition of the compliment: $x \notin (\overline{A} \cap B)$

    By definition of set intersection: $x \notin (\overline{A}\ and\ B)$

    Applying DeMorgan's Law: $x \notin \overline{A}\ or\ x \notin B$

    By definition of the compliment: $x \in A\ or\ x \in \overline{B}$

    By definition of set union: $x \in (A \cup \overline{B})$

    so $x \in A \cup \overline{B}$

    therefore, $\overline{\overline{A} \cup B} = A \cup \overline{B}$

    $\smallskip$

    Suppose $x \in A \cup \overline{B}$

    By definition of set union: $x \in A\ or\ \overline{B}$

    By definition of distribution: $x \in A\ or x \in \overline{B}$

    By definition of the compliment: $x \notin \overline{A}\ or\ x \notin B$

    Applying DeMorgan's Law: $x \notin (\overline{A}\ and\ B)$

    By definition of set intersection: $x \notin (\overline{A} \cap B)$

    By definition of the compliment: $x \in \overline{\overline{A} \cap B}$

    therefore, $A \cup \overline{B} = \overline{\overline{A} \cup B}$

    \newpage

    \question[10]{Collaboration with Sharon Liu and Sahil Kuwadia}

    $(B - A) \cup (C - A) = (B \cup C) - A$

    Suppose $x \in (B - A) \cup (C - A)$

    By definition of set union: $x \in (B - A)\ or\ (C - A)$

    By definition of set difference: $(x \in B\ and\ x \notin A)\ or\ (x \in C\ and\ x \notin A)$

    By definition of distribution: $(x \in B\ or\ x \in C)\ and\ x \notin A$
    
    By definition of distribution: $x \in (B\ or\ C)\ and\ x \notin A$

    By definition of set intersection and set union: $x \in (B \cup C) \cap x \notin A$

    By definition of the compliment: $x \in (B \cup C) \cap x \in \overline{A}$

    By definition of distribution: $x \in (B \cup C) \cap \overline{A}$

    By definition of set difference: $x \in (B \cup C) - A$

    therefore, $(B - A) \cup (C - A) = (B \cup C) - A$

    $\smallskip$

    Suppose $x \in (B \cup C) - A$

    By definition of set difference: $x \in (B \cup C) \cap \overline{A}$

    By definition of distribution: $x \in (B \cup C) \cap x \in \overline{A}$

    By definition of the compliment: $x \in (B \cup C) \cap x \notin A$

    By definition of set intersection and set union: $x \in (B\ or\ C)\ and\ x \notin A$

    By definition of distribution: $(x \in B\ or\ x \in C)\ and\ x \notin A$

    By definition of distribution: $(x \in B\ and\ x \notin A)\ or\ (x \in C\ and\ x \notin A)$

    By definition of set intersection: $(x \in B \cap x \notin A)\ or\ (x \in C \cap x \notin A)$

    By definition of set difference: $x \in (B - A)\ or\ (C - A)$

    By definition of set union $x \in (B - A) \cup (C - A)$

    therefore, $ (B \cup C) - A = (B - A) \cup (C - A)$

\end{document}