\documentclass{article} % This command is used to set the type of document you are working on such as an article, book, or presenation

\usepackage{geometry} % This package allows the editing of the page layout
\usepackage{amsmath}  % This package allows the use of a large range of mathematical formula, commands, and symbols
\usepackage{graphicx}  % This package allows the importing of images
\usepackage{soul}
\usepackage{amsfonts}
\usepackage{dirtytalk}
\usepackage{tabto}
\usepackage{xcolor,colortbl, amssymb}
\usepackage{comment}


\newcommand{\question}[2][]{\begin{flushleft}
        \textbf{Question #1}: \textit{#2}

\end{flushleft}}

\definecolor{Green}{rgb}{0, 1, 0}
\definecolor{Pink}{rgb}{1, .753, .796}

\newcommand{\sol}{\textbf{Solution}:} %Use if you want a boldface solution line
%\newcommand\tab[1][0.4cm]{\hspace*{#1}}
\newcommand{\maketitletwo}[2][]{\begin{center}
        \Large{\textbf{Homework #1}
            
            CMPSC 360} % Name of course here
        \vspace{5pt}
        
        \normalsize{Kinner Parikh  % Your name here
        
        \today}        % Change to due date if preferred
        \vspace{15pt}
        
\end{center}}
\begin{document}
    \maketitletwo[4]  % Optional argument is assignment number
    %Keep a blank space between maketitletwo and \question[1]
    
    \question[1]{}
    
    NEWS\_KIT = I was reading the newspaper in the kitchen \tabto*{11cm} H1: NEWS\_KIT $\rightarrow$ GLASS\_KIT
    
    GLASS\_KIT = My glasses are on the kitchen table \tabto*{11cm} H2: GLASS\_KIT $\rightarrow$ GLASS\_BREAK
    
    GLASS\_BREAK = I saw my glasses at breakfast \tabto*{11cm} H3: $\neg$ GLASS\_BREAK

    NEWS\_LIV = I was reading the newspaper in the living room \tabto*{11cm} H4: NEWS\_LIV $\lor$ NEWS\_KIT
    
    GLASS\_COFF = My glasses are on the coffee table \tabto*{11cm} H5: NEWS\_LIV $\rightarrow$ GLASS\_COFF
    
    \hspace*{0cm}

    1. GLASS\_KIT $\rightarrow$ GLASS\_BREAK \tabto*{7cm}H2

    2. $\neg$GLASS\_BREAK \tabto*{7cm}H3

    3. $\neg$GLASS\_KIT \tabto*{7cm}Modus Tollens on 1 and 2

    4. NEWS\_KIT $\rightarrow$ GLASS\_KIT \tabto*{7cm}H1

    5. $\neg$NEWS\_KIT \tabto*{7cm}Modus Tollens on 4 and 3

    6. NEWS\_LIV $\lor$ NEWS\_KIT \tabto*{7cm}H4

    7. NEWS\_LIV \tabto*{7cm}Disjunctive Syllogism on 6 and 5

    8. NEWS\_LIV $\rightarrow$ GLASS\_COFF \tabto*{7cm}H5

    9. GLASS\_COFF \tabto*{7cm}Modus Ponens on 8 and 7


    \hspace*{0cm}

    Therefore, the glasses are at on the coffee table.

    \question[2]{}

    H1: ($\neg$v $\lor$ $\neg$p) $\rightarrow$ (s $\land$ z)
    
    H2: s $\rightarrow$ o
    
    H3: $\neg$o
    
    C: v

    \hspace*{0cm}

    1. s $\rightarrow$ o \tabto*{5cm}H2

    2. $\neg$o \tabto*{5cm}H3

    3. $\neg$s \tabto*{5cm}Modus Tollens on 1 and 2

    4. ($\neg$v $\lor$ $\neg$p) $\rightarrow$ (s $\land$ z) \tabto*{5cm}H1
    
    5. $\neg$v $\rightarrow$ (s $\land$ z) \tabto*{5cm}Additive rule on 4

    6. $\neg$v $\rightarrow$ s \tabto*{5cm}Simplification of 5

    7. $\neg \neg$v \tabto*{5cm}Modus Tollens of 6 and 3

    8. v \tabto*{5cm}Double negation on 7

    \question[3]{}

    1. This is not a valid argument (a$^2$ is positive, but a could be $\pm$a)
    
    2. This is a valid argument (the only solution for $\sqrt{0^2}$ is 0)

    \question[4]{}

    P($x$) = if $x$ has taken CMPSC-360, then they can take CMPSC-465 next semester

    c = Miranda, Tom, Shilpa, Mark, and Desiree

    U = Every student who has taken CMPSC-360

    \hspace*{0cm}

    \underline{$\forall x$P($x$) \phantom{aaaaaa}}

    $\therefore$P($c$) if $c \in $ U

    \hspace*{0cm}

    The argument is valid because of universal instantiation.

    \newpage




    \question[5]{}

    %For all natural numbers $n$, $\frac{n}{3} + \frac{n^2}{2} + \frac{n^3}{6}$ is a natural number

    Proof:

    Suppose $\frac{n}{3} + \frac{n^2}{2} + \frac{n^3}{6}$, and we know that $n \in \mathbb{N}$

    $\frac{n}{3} + \frac{n^2}{2} + \frac{n^3}{6}$ = $\frac{2n}{6} + \frac{3n^2}{6} + \frac{n^3}{6}$ \tabto*{6cm}using algebra
    
    \vspace*{0.08cm}

    \tabto*{2.57cm} = $\frac{n}{6}(2 + 3n + n^2)$ \tabto*{6cm}simplifying the equation

    \vspace*{0.08cm}

    \tabto*{2.57cm} = $\frac{1}{6}n(n + 1)(n + 2)$ \tabto*{6cm}factoring
    
    \vspace*{0.08cm}

    By definition of divides, $n(n + 1)(n + 2)$ = 6$c$ where $c \in \mathbb{N}$

    A natural number $n$ can either be 1 or $x+1$ where $x \geq 1$

    \vspace*{0.2cm}

    \textbf{Case 1}: $n$ = 1

    \tabto*{1cm}$n(n + 1)(n + 2)$ = $(1)(1 + 1)(1 + 2)$ \tabto*{7.3cm}substitute 1 for $n$

    \vspace*{0.08cm}

    \tabto*{1cm}\tabto*{3.51cm} = $(1)(2)(3)$ \tabto*{7.3cm}addition
    
    \vspace*{0.08cm}

    \tabto*{1cm}\tabto*{3.51cm} = $6$ \tabto*{7.3cm}multiplication
    
    \vspace*{0.08cm}

    \tabto*{1cm}therefore, 6 = 6$c$, which simplifies to c = 1.

    \tabto*{1cm}Since 1 $\in \mathbb{N}$, the case $n$ = 1 is validated

    \vspace*{0.2cm}
    
    \textbf{Case 2}: $n$ = $x + 1$
    \vspace*{0.08cm}

    \tabto*{1cm}$n(n + 1)(n + 2)$ = $(x + 1)\cdot [(x + 1) + 1]\cdot [(x + 1) + 2]$ \tabto*{9.6cm}plugging in $x+1$ for $n$
    \vspace*{0.08cm}

    \tabto*{3.51cm} = $(x + 1)(x + 2)(x + 3)$ \tabto*{9.6cm}addition
    \vspace*{0.08cm}

    \tabto*{3.51cm} = $x^3 + 6x^2 + 11x + 6$ \tabto*{9.6cm}algebra
    \vspace*{0.08cm}

    \tabto*{3.51cm} = $(x^3 +3x^2 + 2x) + (3x^2 + 9x + 6)$ \tabto*{9.6cm}algebra
    \vspace*{0.08cm}

    \tabto*{3.51cm} = $x(x+1)(x+2) + 3(x+1)(x+2)$ \tabto*{9.6cm}factoring
    \vspace*{0.08cm}

    \tabto*{3.51cm} = $6c + 3(x+1)(x+2)$ \tabto*{9.6cm}substitute $x(x+1)(x+2)$ for $6c$
    \vspace*{0.08cm}

    \tabto*{2cm} \textbf{Case 2a}: $x$ is even, such that $x$ = $2a$ for some $a \in \mathbb{N}$
    \vspace*{0.08cm}

    \tabto*{2cm} $6c + 3(x+1)(x+2)$ = $6c + 3(2a + 1)(2a + 2)$ \tabto*{9.5cm}plugging in $2a$ for $x$
    \vspace*{0.08cm}

    \tabto*{5.24cm} = $6c + 3(2a + 1)[2(a + 1)]$ \tabto*{9.5cm}factoring out 2
    \vspace*{0.08cm}

    \tabto*{5.24cm} = $6c + 6(2a + 1)(a + 1)$ \tabto*{9.5cm}multiplication
    \vspace*{0.08cm}
    
    \tabto*{5.24cm} = $6[c + (2a + 1)(a + 1)]$\tabto*{9.5cm}factoring out 6
    \vspace*{0.08cm}

    \tabto*{5.24cm} = $6 \cdot k_{1}$ for some $k_{1} \in \mathbb{N}$ where $k_{1} = [c + (2a + 1)(a + 1)]$

    %\tabto*{2cm} By definition of divides, $c + (2a + 1)(a + 1)$ = $6d$ for some $d \in \mathbb{N}$

    \tabto*{2cm} Therefore, by definition of an even number, $6c + 3(x+1)(x+2)$ is divisible by 6
    \vspace*{0.3cm}

    \tabto*{2cm} \textbf{Case 2b}: $x$ is odd, such that $x$ = $2a + 1$ for some $a \in \mathbb{N}$
    \vspace*{0.08cm}

    \tabto*{2cm} $6c + 3(x+1)(x+2)$ = $6c + 3(2a+1 + 1)(2a+1 + 2)$ \tabto*{10.4cm}plugging in $2a+1$ for $x$
    \vspace*{0.08cm}

    \tabto*{5.24cm} = $6c + 3(2a+2)(2a+3)$ \tabto*{10.4cm}addition
    \vspace*{0.08cm}

    \tabto*{5.24cm} = $6c + 3[2(a+1)](2a+3)$ \tabto*{10.4cm}factoring out 2
    \vspace*{0.08cm}

    \tabto*{5.24cm} = $6c + 6(a+1)(2a+3)$ \tabto*{10.4cm}multiplication
    \vspace*{0.08cm}

    \tabto*{5.24cm} = $6[c + (a+1)(2a+3)]$ \tabto*{10.4cm}factoring out 6
    \vspace*{0.08cm}
    
    \tabto*{5.24cm} = $6 \cdot k_{2}$ for some $k_{2} \in \mathbb{N}$ where $k_{2} = [c + (a+1)(2a+3)]$

    %\tabto*{2cm} By definition of divides, $c + (a + 1)(2a + 3)$ = $6d$ for some $d \in \mathbb{N}$

    \tabto*{2cm} Therefore, by definition of an odd number, $6c + 3(x+1)(x+2)$ is divisible by 6

    \tabto*{1cm}We notice that $n(n + 1)(n + 2)$ where $n$ = $x + 1$ is divisible by 6 in both cases

    We notice that $n(n + 1)(n + 2)$ is divisible by 6 in both cases.

    Therefore, for all natural numbers $n$, $\frac{n}{3} + \frac{n^2}{2} + \frac{n^3}{6}$ is a natural number. $\square$

    \question[6]{}

    i) The assumption is that $n$ is an odd natural number

    ii) We want to conclude that $n^2 -1$ is always divisible by 8

    Proof:

    Suppose $n^2-1$, and n is an odd natural number
    
    By definition of odd, we know that $n = 2a+1$ such that $a \in \mathbb{N}$

    $n^2-1$ = $(2a+1)^2-1$ \tabto*{5cm}plugging in $2a + 1$ for $n$

    \tabto*{1.64cm} = $4a^2 + 4a + 1 -1$ \tabto*{5cm}algebra

    \tabto*{1.64cm} = $4a^2 + 4a$ \tabto*{5cm}subtraction

    \tabto*{1.64cm} = $4a(a + 1)$ \tabto*{5cm}factoring out 4$a$

    A natural number $a$ can be either an even or odd natural number 

    \vspace*{0.2cm}

    \textbf{Case 1}: Suppose $a$ is an even natural number. By definition of even we get $a = 2x; x \in \mathbb{N}$

    $4a(a + 1)$ = $4(2x)(2x + 1)$ \tabto*{5cm}plugging in $2x$ for $a$

    \tabto*{2.08cm} = $8x(2x + 1)$ \tabto*{5cm}multiplication

    \tabto*{2.08cm} = $8\cdot k$ for some $k \in \mathbb{N}$ where $k = x(2x + 1)$

    % By definition of divides, $x(2x + 1)$ = $8c$ for some $c \in \mathbb{N}$

    Therefore, when $a$ is even, the function is divisible by 8

    \vspace*{0.2cm}

    \textbf{Case 2}: Suppose $a$ is an odd natural number. By definition of odd we get $a = 2x + 1; x \in \mathbb{N}$

    $4a(a + 1)$ = $4(2x + 1)(2x + 1+ 1)$ \tabto*{6cm}plugging in $2x + 1$ for $a$

    \tabto*{2.08cm} = $4(2x + 1)(2x + 2)$ \tabto*{6cm}addition

    \tabto*{2.08cm} = $4(2x + 1)[2(x + 1)]$ \tabto*{6cm}addition

    \tabto*{2.08cm} = $8(2x + 1)(x + 1)$ \tabto*{6cm}multiplication

    \tabto*{2.08cm} = $8\cdot k$ for some $k \in \mathbb{N}$ where $k = (2x + 1)(x + 1)$

    %By definition of divides, $(2x + 1)(x+1)$ = $8c$ for some $c \in \mathbb{N}$

    Therefore, when $a$ is odd, the function is divisible by 8

    \vspace*{0.3cm}

    We notice that $4a(a+1)$ is divisible by 8 for both cases.

    Therefore, $n^2 -1$ is always divisible by 8 when $n$ is an odd natural number $\square$

    \question[7]{}

    i) We know that $m \in \mathbb{Z}$

    ii) Assume that $m$ is odd

    iii) Prove $3m + 7$ is even

    \vspace*{0.3cm}
    Proof:

    Suppose $m \in \mathbb{Z}$, and we know that $m$ is odd
    
    By definition of odd, we know that $m = 2a+1$ such that $a \in \mathbb{Z}$

    $3m + 7$ = $3(2a + 1) + 7$ \tabto*{5cm}plugging in $2a + 1$ for $m$

    \tabto*{1.75cm} = $6a + 3 + 7$ \tabto*{5cm}multiplication

    \tabto*{1.75cm} = $6a + 10$ \tabto*{5cm}addition

    \tabto*{1.75cm} = $2(3a + 5)$ \tabto*{5cm}factoring out 2

    \tabto*{1.75cm} = $2\cdot k$ for some $k \in \mathbb{Z}$ where $k = 3a + 5$

    Therefore, by the definition of an even number, $3m + 7$ is even when $m$ is odd. $\square$

    \newpage

    \question[8]{}

    Prove that the product of any two odd integers is odd.

    Proof:

    Suppose $a \in \mathbb{Z}$ and $b \in \mathbb{Z}$ such that $a$ and $b$ are odd

    By definition of odd, $a = 2x + 1$ such that $x \in \mathbb{Z}$ and $b = 2y + 1$ such that $y \in \mathbb{Z}$

    so, $a \cdot b$ = $(2x + 1)(2y + 1)$ substituting $a$ for $2x + 1$ and $b$ for $2y + 1$

    \tabto*{1.78cm} = $4xy + 2x + 2y + 1$ multiplication

    \tabto*{1.78cm} = $2(2xy + x + y) + 1$ factoring out 2

    \tabto*{1.78cm} = $2k + 1$ for some $k \in \mathbb{Z}$ where $k = 2xy + x + y$

    Therefore, by the definition of an odd number, $a \cdot b$ is odd

    Therefore, the product of any two odd integers is odd. $\square$



    
\end{document}