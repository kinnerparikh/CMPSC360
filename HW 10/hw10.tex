\documentclass{article} % This command is used to set the type of document you are working on such as an article, book, or presenation

\usepackage{geometry} % This package allows the editing of the page layout
\usepackage{amsmath}  % This package allows the use of a large range of mathematical formula, commands, and symbols
\usepackage{graphicx}  % This package allows the importing of images
\usepackage{soul}
\usepackage{amsfonts}
\usepackage{dirtytalk}
\usepackage{tabto}
\usepackage{xcolor,colortbl, amssymb}

% https://www.messletters.com/en/big-text/

\newcommand{\question}[2][]{\begin{flushleft}
        \textbf{Question #1}: #2

\end{flushleft}}

\definecolor{Green}{rgb}{0, 1, 0}
\definecolor{Pink}{rgb}{1, .753, .796}

\newcommand{\sol}{\textbf{Solution}:} %Use if you want a boldface solution line
%\newcommand\tab[1][0.4cm]{\hspace*{#1}}
\newcommand{\maketitletwo}[2][]{\begin{center}
        \Large{\textbf{Homework #1}
            
            CMPSC 360} % Name of course here
        \vspace{5pt}
        
        \normalsize{Kinner Parikh  % Your name here
        
        \today}        % Change to due date if preferred
        \vspace{15pt}
        
\end{center}}
\begin{document}
    \maketitletwo[10]  % Optional argument is assignment number
    %Keep a blank space between maketitletwo and \question[1]
    
    \question[1]{Solve the congruence $8x \equiv 13$ mod 29}

    Finding $c^{-1}:$
    \begin{alignat*}{1}
        29 &= 8 \cdot 3 + 5\\
        8  &= 5 \cdot 1 + 3\\
        5  &= 3 \cdot 1 + 2\\
        3  &= 2 \cdot 1 + 1\\
        2  &= 1 \cdot 2\\
        \\
        1 &= 3 - 2 \cdot 1\\
          &= 3 - (5 - 3)\\
          &= -5 + 3 \cdot 2\\
          &= -5 + (8 - 5) \cdot 2\\
          &= 8 \cdot 2 - 5 \cdot 3\\
          &= 8 \cdot 2 - (29 - 8 \cdot 3) \cdot 3\\
          &= 29 \cdot (-3) + 8 \cdot 11
    \end{alignat*}

    So, $c^{-1} = 11$

    Multiplying both sides of congruence by $c^{-1}$:
    \begin{alignat*}{2}
        8 \cdot 11x &\equiv 13 \cdot 11 \text{ mod } 29\ \\
        x &\equiv 143 \text{ mod } 29 && \text{[since 8 $\cdot$ 11 mod 29 = 1]}\\
        x &\equiv 143 = 27 \text{ mod } 29\ && \text{[since 143 mod 29 = 27]}
    \end{alignat*}

    So a possible value for $x$ is 27.

    \newpage

    \question[2]{Solve the congruence $55x = 34$ (mod 89) and find all possible values of $x$}

    Finding the inverse 55 mod 89:
    \begin{alignat*}{1}
        89 &= 55 \cdot 1 + 34\\
        55 &= 34 \cdot 1 + 21\\
        34 &= 21 \cdot 1 + 13\\
        21 &= 13 \cdot 1 + 8\\
        13 &= 8 \cdot 1 + 5\\
        8 &= 5 \cdot 1 + 3\\
        5 &= 3 \cdot 1 + 2\\
        3 &= 2 \cdot 1 + 1\\
        2 &= 1 \cdot 2\\
        \\
        1 &= 3 - 2\\
        &= 3 - (5 - 3)\\
        &= -5 + 3 \cdot 2\\
        &= -5 + (8 - 5) \cdot 2\\
        &= 8 \cdot 2 + 5 \cdot (-3)\\
        &= 8 \cdot 2 + (13 - 8) \cdot (-3)\\
        &= 13 \cdot (-3) + 8 \cdot 5\\
        &= 13 \cdot (-3) + (21 - 13) \cdot 5\\
        &= 21 \cdot 5 + 13 \cdot (-8)\\
        &= 21 \cdot 5 + (34 - 21) \cdot (-8)\\
        &= 34 \cdot (-8) + 21 \cdot 13\\
        &= 34 \cdot (-8) + (55 - 34) \cdot 13\\
        &= 55 \cdot 13 + 34 \cdot (-21)\\
        &= 55 \cdot 13 + (89 - 55) \cdot (-21)\\
        &= 89 \cdot (-21) + 55 \cdot 34
    \end{alignat*}

    So $c^{-1} = 34$
    Multiplying both sides of congruence by $c^{-1}$:
    \begin{alignat*}{2}
        55 \cdot 34x &\equiv 34 \cdot 34 \text{ mod } 89\ \\
        x &\equiv 1156 \text{ mod } 89 && \text{[since 55 $\cdot$ 34 mod 89 = 1]}\\
        x &\equiv 1156 = 88 \text{ mod } 89\ && \text{[since 143 mod 89 = 27]}
    \end{alignat*}

    So, $x = 88 + 89k$ where $k \in \mathbb{Z}$ satisfies the congruence form: $55x = 34$ (mod 89)

    \newpage

    \question[3]{}

    $z_2 = 105 / 7 = 15$

    $y_2 \cdot 15 = 1 \text{ mod } 7 \rightarrow y_2 = 1$ 

    $(7 \cdot 11 \cdot 7) + (4 \cdot 10 \cdot 15) + (6 \cdot 9 \cdot 9) = 1625$

    $x = 1625 \text{ mod } 105 = 50$

    \question[4]{Using Fermat's Little Theorem find $3^{2003}$ mod 455}

    The prime factorization of 455 is 5, 7, 13

    \textbf{Part 1}: $3^{2003}$ mod 5

    \tabto{1cm} We know that $3^4 \equiv 1$ mod 5

    \tabto{1cm} $2003 = 4 \cdot 500 + 3$

    \tabto{1cm} $3^{2003}$ mod 5 = $3^{4 \cdot 500} \cdot 3^{3}$ mod 5

    \tabto{1cm} $1 \cdot 3^{3}$ mod 5 = 27 mod 5 = 2 mod 5

    \textbf{Part 2}: $3^{2003}$ mod 7

    \tabto{1cm} We know that $3^6 \equiv 1$ mod 7

    \tabto{1cm} $2003 = 333 \cdot 6 + 5$

    \tabto{1cm} $3^{2003}$ mod 7 = $3^{6 \cdot 333} \cdot 3^{5}$ mod 7

    \tabto{1cm} $1 \cdot 3^{5}$ mod 7 = 243 mod 7 = 5 mod 7

    \textbf{Part 3}: $3^{2003}$ mod 13

    \tabto{1cm} We know that $3^3 \equiv 1$ mod 13

    \tabto{1cm} $2003 = 667 \cdot 3 + 2$

    \tabto{1cm} $3^{2003}$ mod 7 = $3^{3 \cdot 667} \cdot 3^{2}$ mod 13

    \tabto{1cm} $1 \cdot 3^{2}$ mod 13 = 9 mod 13

    \vspace*{0.1cm}

    $x$ = 2 mod 5

    $x$ = 5 mod 7

    $x$ = 9 mod 13

    Applying the Chinese Remainder Theorem:

    $a_1 = 2, a_2 = 5, a_3 = 9$ and $m_1 = 5, m_2 = 7, m_3 = 13$

    So, $M = 5 \cdot 7 \cdot 13 = 455$

    Thus, $z_1 = 91, z_2 = 65, z_3 = 35$

    $y_1 \cdot 91 = $ 1 mod 5; so $y_1$ = 1

    $y_2 \cdot 65 = $ 1 mod 7; so $y_2$ = 4

    $y_3 \cdot 35 = $ 1 mod 13; so $y_3$ = 3

    We get $x = (2 \cdot 91 \cdot 1) + (5 \cdot 65 \cdot 4) + (9 \cdot 35 \cdot 3)$ = 2427

    And 2427 mod 455 = 152

    Thus, $3^{2003}$ mod 455 = 152

    \newpage
    
    \question[5]{}

    TIME FOR FUN

    \question[6]{We chose two prime numbers $p = 17$, $q = 11$, and $e = 7$. Calculate $d$ and show the 
    public and private keys.}

    $n = pq = 17 \cdot 11 = 187$
    
    $k = (p-1)(q-1) = 16 \cdot 10 = 160$

    $de \equiv 1 (\text{mod } 160)$, so $d \cdot 7 \equiv 1 (\text{mod } 160)$
    \begin{alignat*}{1}
        160 &= 7 \cdot 22 + 6\\
        7 &= 6 \cdot 1 + 1\\
        6 &= 1 \cdot 6\\
        \\
        1 &= 7 - 6\\
        &= 7 - (160 - 7 \cdot 22)\\
        &= -160 + 7 \cdot 23\\
    \end{alignat*}

    So, we know that $d = 23$

    The public key is: (187, 7)

    The private key is: (187, 23)

    \newpage

    \question[7]{Given $p = 37$ and $q = 43$, can we choose $d = 71$? If yes, justify your answer, 
    otherwise suggest one value for $d$. Then compute the public and the private keys.}

    $n = pq = 37 \cdot 43 = 1591$

    $k = (p-1)(q-1) = 36 \cdot 42 = 1512$

    Finding the inverse of 71 mod 1512:
    \begin{alignat*}{1}
        1512 &= 71 \cdot 21 + 21\\
        71   &= 21 \cdot 3 + 8\\
        21   &= 8 \cdot 2 + 5\\
        8    &= 5 \cdot 1 + 3\\
        5    &= 3 \cdot 1 + 2\\
        3    &= 2 \cdot 1 + 1\\
        2    &= 1 \cdot 2\\
        \\
        1 &= 3 - 2\\
          &= 3 - (5 - 3)\\
          &= -5 + 3 \cdot 2\\
          &= -5 + (8 - 5) \cdot 2\\
          &= 8 \cdot 2 + 5 \cdot (-3)\\
          &= 8 \cdot 2 + (21 - 8 \cdot 2) \cdot (-3)\\
          &= 21 \cdot (-3) + 8 \cdot 8\\
          &= 21 \cdot (-3) + (71 - 21 \cdot 3) \cdot 8\\
          &= 71 \cdot 8 + 21 \cdot (-27)\\
          &= 71 \cdot 8 + (1512 - 71 \cdot 21) \cdot (-27)\\
          &= 1512 \cdot (-27) + 71 \cdot 575
    \end{alignat*}

    The inverse of 71 mod 1512 is 575. So $e = 575$ 

    We must calculate gcd(575, 1512)
    \begin{alignat*}{1}
        1512 &= 575 \cdot 2 + 362\\
        575 &= 362 \cdot 1 + 213\\
        362 &= 213 \cdot 1 + 149\\
        213 &= 149 \cdot 1 + 64\\
        149 &= 64 \cdot 2 + 21\\
        64 &= 21 \cdot 3 + 1\\
        21 &= 1 \cdot 21
    \end{alignat*}

    So gcd(575, 1512) = 1, which means we can choose $d = 71$

    Public key: (1591, 575)

    Private key: (1591, 71)

    \question[8]{}

    $2x \equiv 5$(mod 7)
    
    Applying the backwards pass of Euclid division, we know that 2 inverse of mod 7 is 4.

    Multiplying both sides of congruence:

    $2(4)x \equiv 5(4)$(mod 7)

    $x \equiv 20$(mod 7); Since $2 \cdot 4 \equiv 1$(mod 7)

    So, $x \equiv 6$(mod 7)

    \vspace{0.2cm}

    $4x \equiv 2$(mod 6)

    Dividing congruence by 2, we get $2x \equiv 1$(mod 3)

    Applying the backwards pass of Euclid division, we know that 2 inverse of mod 3 is 2.

    $2(2)x \equiv 1(2)$(mod 3)

    $x \equiv 2$(mod 3); Since $2 \cdot 2 \equiv 1$(mod 3)

    So, $x \equiv 2$(mod 3)

    \vspace{0.2cm}
    
    $x \equiv 2$(mod 3)
    
    $x \equiv 3$(mod 5)
    
    $x \equiv 6$(mod 7)

    Applying the Chinese Remainder Theorem:

    $a_1 = 2, a_2 = 3, a_3 = 6$ and $m_1 = 3, m_2 = 5, m_3 = 7$

    So, $M = 3 \cdot 5 \cdot 7 = 105$

    Thus, $z_1 = 35, z_2 = 21, z_3 = 15$

    $y_1 \cdot 35 = 1$ mod 3; so $y_1 = 2$

    $y_2 \cdot 21 = 1$ mod 5; so $y_2 = 1$

    $y_3 \cdot 15 = 1$ mod 7; so $y_3 = 1$

    We get $x = (2 \cdot 35 \cdot 2) + (3 \cdot 21 \cdot 1) + (6 \cdot 15 \cdot 1) = 293$

    And 293 mod 105 = 83

    Thus, the lowest possible simultaneous solution is $x = 83$

    \newpage

    \question[9a]{}

    Proof:

    We have to find that for every polynomial of degree $n$ with integer coefficients $f(x)$, 
    
    we have $f (b_1) \equiv f (b_2)$ mod $p$.
    
    We must prove that for every term like $a_k x^k$ in the polynomial $f (x)$ this property holds
    
    Assume $b_ 1 \equiv b_2 \text{ mod }p$ such that $b_1, b_2, p \in \mathbb{Z}$

    This means that, from the definition of congruence modulo, $p \mid (b_1 - b_2)$

    \textbf{Case 1}: $b_1 + c \equiv b_2 + c \text{ mod } p$ for arbitrary integer $c$

    \tabto{1cm} From the definition of congruence modulo, $p \mid [b_1 + c - (b_2 + c)]$ = $p \mid (b_1 - b_2)$

    \tabto{1cm} So taking the reverse of the definition of congruence modulo, we get $b_1 \equiv b_2 \text{ mod } p$

    \tabto{1cm} Thus, $b_1 \equiv b_2 \text{ mod } p \Rightarrow b_1 + c \equiv b_2 + c \text{ mod } p$ for arbitrary integer $c$

    \textbf{Case 2}: $c \cdot b_1 \equiv c \cdot b_2 \text{ mod } p$ for arbitrary integer $c$

    \tabto{1cm} From the definition of congruence modulo, $p \mid (c \cdot b_1 - c \cdot b_2)$ = $p \mid c \cdot (b_1 - b_2)$

    \tabto{1cm} We know that as a property of division, if $a \mid b$, then $a \mid bt$. Similarly, we already know that 
    
    \tabto{1cm} $p \mid (b_1 - b_2)$, so we can say that $p \mid c \cdot (b_1 - b_2)$

    \tabto{1cm} Thus, we know that $p \mid c \cdot (b_1 - b_2) = p \mid (b_1 - b_2)$
    
    \tabto{1cm} So taking the reverse of the definition of congruence modulo, we get $b_1 \equiv b_2 \text{ mod } p$

    \tabto{1cm} Thus, $b_1 \equiv b_2 \text{ mod } p \Rightarrow c \cdot b_1 \equiv c \cdot b_2 \text{ mod } p$ for arbitrary integer $c$

    \textbf{Case 3}: ${b_1}^k \equiv {b_2}^k \text{ mod } p$ for a positive integer $k$.

    \tabto{1cm} From the definition of congruence modulo, $p \mid ({b_1}^k - {b_2}^k)$

    \tabto{1cm} We proceed by induction on $k$

    \tabto{1cm} \textbf{Base Case:} $(k = 1)$

    \tabto{1.5cm} So $b_1 \equiv b_2 \text{ mod } p$, which we already know is true.

    \tabto{1.5cm} The base case is proven

    \tabto{1cm} \textbf{Inductive Hypothesis:} $(k = n)$

    \tabto{1.5cm} For an arbitrary positive integer $n$, assume that ${b_1}^n \equiv {b_2}^n \text{ mod } p$

    \tabto{1cm} \textbf{Inductive Step:} $(k = n + 1)$

    \tabto{1.5cm} We have to show that ${b_1}^{n + 1} \equiv {b_2}^{n + 1} \text{ mod } p$

    \tabto{1.5cm} Expanding both sides: $b_1 \cdot {b_1}^n \equiv b_2 \cdot {b_2}^n \text{ mod } p$

    \tabto{5.07cm} $b_1 \cdot {b_1}^n \equiv (b_2 \text{ mod } p \cdot {b_2}^n \text{ mod } p) \text{ mod } p$

    \tabto{1.5cm} From the base case and inductive hypothesis, we can say that ${b_1}^{n + 1} \equiv {b_2}^{n + 1} \text{ mod } p$

    \tabto{1cm} Therefore, we can say that $\forall k \in \mathbb{Z}^+;\ {b_1}^k \equiv {b_2}^k \text{ mod } p$

    Therefore, we can say that $f({b_1}) \equiv f({b_2}) \text{ mod } p. \ \square$

    \newpage

    \question[9b]{}

    Proof:

    Assume that we are in the decimal number system, $z \in \mathbb{Z}$, and $9 \mid z$.

    We can write $z$ as $(a_k a_{k - 1} a_{k - 2} ... a_0)$, where $a_k$ represents a digit of $z$

    Using the conclusion we reached in the previous question, we know that:
    
    \tabto{4cm} $f(x) = a_kx^k + a_{k - 1}x^{k - 1} + ... + a_0$

    $x$ represents the base of the number system. 

    So we can say that $f(10) = z$, and because of this, we know that $b_1 \equiv b_2 \text{ mod } p$

    Thus we can say that $(n - 1) \mid z$





\end{document}