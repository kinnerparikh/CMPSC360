\documentclass{article} % This command is used to set the type of document you are working on such as an article, book, or presenation

\usepackage{geometry} % This package allows the editing of the page layout
\usepackage{amsmath}  % This package allows the use of a large range of mathematical formula, commands, and symbols
\usepackage{graphicx}  % This package allows the importing of images
\usepackage{soul}
\usepackage{amsfonts}
\usepackage{dirtytalk}
\usepackage{tabto}
\usepackage{xcolor,colortbl, amssymb}

% https://www.messletters.com/en/big-text/

\newcommand{\question}[2][]{\begin{flushleft}
        \textbf{Question #1}: #2

\end{flushleft}}

\definecolor{Green}{rgb}{0, 1, 0}
\definecolor{Pink}{rgb}{1, .753, .796}

\newcommand{\sol}{\textbf{Solution}:} %Use if you want a boldface solution line
%\newcommand\tab[1][0.4cm]{\hspace*{#1}}
\newcommand{\maketitletwo}[2][]{\begin{center}
        \Large{\textbf{Quiz #1}
            
            CMPSC 360} % Name of course here
        \vspace{5pt}
        
        \normalsize{Kinner Parikh  % Your name here
        
        \today}        % Change to due date if preferred
        \vspace{15pt}
        
\end{center}}
\begin{document}
    \maketitletwo[7]  % Optional argument is assignment number
    %Keep a blank space between maketitletwo and \question[1]
    
    \question[2]{Consider the statement $5 \mid (n^5 - n)$ for all $n \geq 0$}

    Proof:

    We proceed by induction on the variable $n$.

    Let $P(n)$ hold the property statement for $n$.

    \textbf{Base Case} $(n = 0)$:

    We need to prove $5 \mid (0^5 - 0)$.

    By definition of divides, we get $0^5 - 0 = 5a$ for some $a \in \mathbb{Z}$.

    We get $0 - 0 = 0 = 5 \cdot 0 = 0$

    The base case is proved.

    \textbf{Inductive Hypothesis} $(n = k)$:

    For any arbitrary integer $n = k$ where $k \geq 0$, assume that $P(k)$ is true

    This means $5 \mid (k^5 - k)$

    Using the definition of divides, we get $k^5 - k = 5q$ where $q \in \mathbb{Z}$.

    \textbf{Inductive Step} $(n = k + 1)$:

    We have to show that $P(k + 1)$ is true, which means $5 \mid [(k + 1)^5 - (k + 1)]$

    Expanding the expression, we get:
    \begin{alignat*}{2}
        (k + 1)^5 - (k + 1) &= k^5+5k^4+10k^3+10k^2+5k+1 - k - 1\ &&\text{[algebra]}\\        
        &= k^5+5k^4+10k^3+10k^2+4k\\
        &= (k^5 - k) +5k^4+10k^3+10k^2+5k\\
        &= 5q+5k^4+10k^3+10k^2+5k &&\text{[inductive step]}\\
        &= 5(q + k^4 + 2k^3 + 2k^2 + k)\\
        &= 5t\ \text{for some $t \in \mathbb{Z}$ and $t = q + k^4 + 2k^3 + 2k^2 + k$}\\
    \end{alignat*}

    We have $(k + 1)^5 - (k + 1) = 5t$. By definition of divides, we get $5 \mid (k + 1)^5 - (k + 1)$
    
    Therefore, it is true that $\forall n \in \mathbb{Z}, n \geq 0, 5 \mid (n^5 - n).\ \square$
    
    \newpage

    \question[3]{$S_n = S_{n - 1} + S_{n - 2} + S_{n - 3}$, where $n > 2$. $S_0 = 0, S_1 = 1, S_2 = 1$. Prove $\forall n \geq 0, S_n < 2^n$}

    Proof:

    We proceed by strong induction on $n$.

    Let $S_n$ be defined as $S_n = S_{n - 1} + S_{n - 2} + S_{n - 3}$ for $n > 2$ where $S_0 = 0, S_1 = 1, S_2 = 1$

    Let $P(n)$ be the proposition that $S_n < 2^n$. 

    \textbf{Base Case} $(n = 0, n = 1, n = 2, n = 3)$:

    When $n = 0$, we know that $0 < 2 ^ 0 = 0 < 1$. $P(n)$ holds for $n = 0$.

    When $n = 1$, we know that $1 < 2 ^ 1 = 1 < 2$. $P(n)$ holds for $n = 1$.

    When $n = 2$, we know that $1 < 2 ^ 2 = 1 < 4$. $P(n)$ holds for $n = 2$.

    When $n = 3$, we know that $(1 + 1 + 0) < 2 ^ 3 = 2 < 8$. $P(n)$ holds for $n = 3$.

    The base case is proved.

    \textbf{Inductive Hypothesis} $(n = k)$:

    Suppose $k$ is an arbitrary integer greater than $2$. Assume that $P(i)$ is true for all $1 \leq i \leq k$ for 
    
    some integer $k$.

    \textbf{Inductive Step} $(n = k + 1)$:

    We have to prove $P(k + 1)$ is true. This means we have to show that $S_{k + 1} < 2^{k + 1}$.

    Let's explore both sides of the equation:
    \begin{alignat*}{1}
        S_{k + 1} &< 2^{k + 1}\\
        S_{(k + 1) - 1} + S_{(k + 1) - 2} + S_{(k + 1) - 3} &< 2^{k + 1}\\
        S_{k} + S_{k - 1} + S_{k - 2} &< 2^{k + 1}\\
        S_{k} + S_{k - 1} + S_{k - 2} &< 2 \cdot 2^{k}\\
        S_{k} + S_{k - 1} + S_{k - 2} + S_{k - 3} - S_{k - 3} &< 2 \cdot 2^{k}\\
        S_{k} + S_k - S_{k - 3} &< 2 \cdot 2^{k}\ \text{[inductive step]}\\
        2S_{k}- S_{k - 3} &< 2 \cdot 2^{k}\\
    \end{alignat*}

    Since we know that $P(k)$ is true, we can say for sure that $2S_k < 2 \cdot 2^k$.

    This means that it is true that $ 2S_{k}- S_{k - 3} < 2 \cdot 2^{k}$.

    Therefore, $\forall n \geq 0, S_n < 2^n.\ \square$
\end{document}