\documentclass{article} % This command is used to set the type of document you are working on such as an article, book, or presenation

\usepackage{geometry} % This package allows the editing of the page layout
\usepackage{amsmath}  % This package allows the use of a large range of mathematical formula, commands, and symbols
\usepackage{graphicx}  % This package allows the importing of images
\usepackage{soul}
\usepackage{amsfonts}
\usepackage{dirtytalk}
\usepackage{tabto}
\usepackage{xcolor,colortbl, amssymb}

% https://www.messletters.com/en/big-text/

\newcommand{\question}[2][]{\begin{flushleft}
        \textbf{Question #1}: #2

\end{flushleft}}

\definecolor{Green}{rgb}{0, 1, 0}
\definecolor{Pink}{rgb}{1, .753, .796}

\newcommand{\sol}{\textbf{Solution}:} %Use if you want a boldface solution line
%\newcommand\tab[1][0.4cm]{\hspace*{#1}}
\newcommand{\maketitletwo}[2][]{\begin{center}
        \Large{\textbf{Class Quiz #1}
            
            CMPSC 360} % Name of course here
        \vspace{5pt}
        
        \normalsize{Kinner Parikh  % Your name here
        
        \today}        % Change to due date if preferred
        \vspace{15pt}
        
\end{center}}
\begin{document}
    \maketitletwo[8]  % Optional argument is assignment number
    %Keep a blank space between maketitletwo and \question[1]

    \question[1]{}

    1) $52^3 \cdot 10^4$

    2) $26^2 \cdot 10^4$

    3) $25 \cdot 26^2 \cdot 10^3$


    
    \question[2]{Find $x$ if $x \equiv 2(\text{mod } 3); x \equiv 2(\text{mod } 4);  x \equiv 1(\text{mod } 5)$ is the simultaneous system of 
    linear congruence using Chinese remainder theorem.}

    $x \equiv 2$ (mod 3)
    
    $x \equiv 2$ (mod 4)
    
    $x \equiv 1$ (mod 5)

    Applying Chinese Remainder Theorem:

    $a_1 = 2, a_2 = 2, a_3 = 1$ and $m_1 = 3, m_2 = 4, m_3 = 5$

    So, $M = 3 \cdot 4 \cdot 5 = 60$

    Thus, $z_1 = 20, z_2 = 15, z_3 = 12$

    $y_1 \cdot 20 = 1$ mod 3; so $y_1 = 2$

    $y_2 \cdot 15 = 1$ mod 4; so $y_2 = 3$

    $y_3 \cdot 12 = 1$ mod 5; so $y_3 = 3$

    We get $x = (2 \cdot 20 \cdot 2) + (2 \cdot 15 \cdot 3) + (1 \cdot 12 \cdot 3) = 206$

    And 206 mod 60 = 26

    Thus, the lowest possible simultaneous solution is $x = 26$

    \question[3]{}

    Bob wants to set up an RSA key pair. He first chooses $p=29$ and $q=31$. Then, $n = 899$

    Also, Bob chooses a valid $e$ from $\{3356, 3357, 3358, 3359\}$ for encryption. $e = 3359$

    Then, Bob encrypts two 3-digits decimal numbers 072, 073 into $m=462, 234$

    Alice's pair of key to decrypt $(n, d)=(n,839)$
    
\end{document}