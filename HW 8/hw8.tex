\documentclass{article} % This command is used to set the type of document you are working on such as an article, book, or presenation

\usepackage{geometry} % This package allows the editing of the page layout
\usepackage{amsmath}  % This package allows the use of a large range of mathematical formula, commands, and symbols
\usepackage{graphicx}  % This package allows the importing of images
\usepackage{soul}
\usepackage{amsfonts}
\usepackage{dirtytalk}
\usepackage{tabto}
\usepackage{xcolor,colortbl, amssymb}

% https://www.messletters.com/en/big-text/

\newcommand{\question}[2][]{\begin{flushleft}
        \textbf{Question #1}: #2

\end{flushleft}}

\definecolor{Green}{rgb}{0, 1, 0}
\definecolor{Pink}{rgb}{1, .753, .796}

\newcommand{\sol}{\textbf{Solution}:} %Use if you want a boldface solution line
%\newcommand\tab[1][0.4cm]{\hspace*{#1}}
\newcommand{\maketitletwo}[2][]{\begin{center}
        \Large{\textbf{Homework #1}
            
            CMPSC 360} % Name of course here
        \vspace{5pt}
        
        \normalsize{Kinner Parikh  % Your name here
        
        \today}        % Change to due date if preferred
        \vspace{15pt}
        
\end{center}}
\begin{document}
    \maketitletwo[8]  % Optional argument is assignment number
    %Keep a blank space between maketitletwo and \question[1]

    %   ___                         _     _                     _ 
    %  / _ \   _   _    ___   ___  | |_  (_)   ___    _ __     / |
    % | | | | | | | |  / _ \ / __| | __| | |  / _ \  | '_ \    | |
    % | |_| | | |_| | |  __/ \__ \ | |_  | | | (_) | | | | |   | |
    %  \__\_\  \__,_|  \___| |___/  \__| |_|  \___/  |_| |_|   |_|
    
    \question[1]{For all $n \in \mathbb{N}$: $3 \mid 2^{2n} - 1$}

    Proof:

    We proceed by induction on the variable $n$.
    
    Let $P(n)$ hold the property of the statement for $n$.

    \textbf{Base Case} (n = 1):

    We need to prove $3 | 2^{2(1)} - 1$

    $2 ^ 2 - 1 = 4 - 1 = 3$

    Since $3 \mid 3$, the base case is proved.

    \textbf{Inductive Hypothesis} $(n = k)$:

    For any arbitrary natural number $n = k$, assume that $P(k)$ is true.

    That means $3 \mid 2^{2k} - 1$

    Using the definition of divides, we get $2^{2k} - 1 = 3q$ where $q \in \mathbb{Z}$

    \textbf{Inductive Step} $(n = k + 1)$:

    We have to show that $P(k + 1)$ is true, which means $3 \mid 2^{2k + 2} - 1$

    Expanding the expression, we get:
    \begin{alignat*}{2}
        2^{2k + 2} - 1 &= 4 \cdot 2^{2k} - 1 \\
        &= 4 \cdot (2^{2k} - 1) + 3 \\
        &= 4 \cdot 3q + 3 && \text{[inductive hypothesis]}\\
        &= 3(4q + 1) && \text{[factoring out 3]}\\
        &= 3t && \text{such that $t \in \mathbb{N}$ where $t = 4q + 1$}
    \end{alignat*}

    Therefore, by definition of divides, $\forall n \in \mathbb{N}, 3 \mid 2^{2n} - 1$. $\square$

    \newpage

    %   ___                         _     _                     ____  
    %  / _ \   _   _    ___   ___  | |_  (_)   ___    _ __     |___ \ 
    % | | | | | | | |  / _ \ / __| | __| | |  / _ \  | '_ \      __) |
    % | |_| | | |_| | |  __/ \__ \ | |_  | | | (_) | | | | |    / __/ 
    %  \__\_\  \__,_|  \___| |___/  \__| |_|  \___/  |_| |_|   |_____|
                                                                   

    \question[2]{Show that $n! > 3^n$ for $n \geq 7$}

    Proof: 

    We proceed by induction on the variable $n$.

    \textbf{Base Case} $(n = 7)$:

    We need to prove $7! > 3^7$

    The left hand side of the equation is 5040 and the right hand side is 2187. Since $5040 > 2187$, 
    
    the base case is proved.

    \textbf{Inductive Hypothesis} $(n = k)$:

    For any arbitrary natural number $n = k$ where $k \geq 7$, we assume that $k! > 3^k$

    \textbf{Inductive Step} $(n = k + 1)$:

    We have to show that $(k + 1)! > 3^{k + 1}$

    To show this, let's explore both sides of the equation

    Expanding both sides we get: $(k + 1) \cdot k! > 3 \cdot 3^k$ 

    From the inductive hypothesis, we know that $k! > 3^k$.

    We also know that $k + 1 > 3$ because of the restriction on $k$ that states $k \geq 7$.

    So, we can conclude that $(k + 1) \cdot k! > 3 \cdot 3^k$, which means $(k + 1)! > 3^{k + 1}$ is true.

    Therefore, $\forall n \in \mathbb{N}, k! > 3^k$. $\square$


    \newpage


    %   ___                         _     _                     _____ 
    %  / _ \   _   _    ___   ___  | |_  (_)   ___    _ __     |___ / 
    % | | | | | | | |  / _ \ / __| | __| | |  / _ \  | '_ \      |_ \ 
    % | |_| | | |_| | |  __/ \__ \ | |_  | | | (_) | | | | |    ___) |
    %  \__\_\  \__,_|  \___| |___/  \__| |_|  \___/  |_| |_|   |____/ 

    \question[3]{For any positive integer $n$, $5 \mid 6^n - 1$}

    Proof:

    We proceed by induction on the variable $n$.

    Let $P(n)$ hold the property of the statement for $n$.

    \textbf{Base Case} $(n = 1)$:

    $P(1)$ asserts that $5 \mid 6^1 - 1$.

    By the definition of divides, $(6^1 - 1) = 5a$ for some $a \in \mathbb{Z}$

    We get, $5 = 5 \cdot 1 = 5$.

    The base case is proved.

    \textbf{Inductive Hypothesis} $(n = k)$:

    For any arbitrary integer $n = k$ where $k \geq 1$, assume that $P(k)$ is true.

    That means $5 \mid 6^k - 1$

    Using the definition of divides, we get $6^k - 1 = 5q$ where $q \in \mathbb{Z}$

    \textbf{Inductive Step} $(n = k + 1)$:

    We have to show that $P(k + 1)$ is true, which means $5 \mid 6^{k + 1} - 1$.

    Expanding the expression, we get:
    \begin{alignat*}{2}
        6^{k + 1} - 1 &= 6 \cdot 6^k - 1 \\
        &=6 \cdot (6^k - 1) + 5\\
        &=6 \cdot 5q + 5&& \text{[inductive step]}\\
        &=5 \cdot (6q + 1)&& \text{[factoring 5 out]} \\
        &=5t&& \text{for some $t \in \mathbb{Z}$ where $t = 6q + 1$}
    \end{alignat*}

    We have  $6^{k + 1} - 1 = 5t$. By definition of divides we get  $5 \mid 6^{k + 1} - 1$.

    Therefore, it is true that $\forall n \in \mathbb{Z}, 5 \mid 6^n - 1.\ \square$ 

    \newpage

    %   ___                         _     _                     _  _   
    %  / _ \   _   _    ___   ___  | |_  (_)   ___    _ __     | || |  
    % | | | | | | | |  / _ \ / __| | __| | |  / _ \  | '_ \    | || |_ 
    % | |_| | | |_| | |  __/ \__ \ | |_  | | | (_) | | | | |   |__   _|
    %  \__\_\  \__,_|  \___| |___/  \__| |_|  \___/  |_| |_|      |_|  
                                                                   

    \question[4]{For any $n \in \mathbb{N}$ and any $a \in \mathbb{R}$, prove that $1 + a + a^2 + a^3 + ... + a^n = \frac{a^{n + 1} - 1}{a - 1}$}

    Proof:

    We proceed by induction on the variable $n$.

    Let $P(n)$ hold the property of the statement for $n$.

    \textbf{Base Case} ($n$ = 1):

    $P(1)$ asserts that $1 + a = \frac{a^{1 + 1} - 1}{a-1}$

    Taking the right hand side:
    \begin{alignat*}{2}
        \frac{a^{1 + 1} - 1}{a-1} &= \frac{a^2 - 1}{a-1}\\
        &= \frac{(a+1)(a-1)}{a-1}\ && [\text{factoring}]\\
        &= a + 1&& [\text{divide}]
    \end{alignat*}

    The base case is proved.

    \textbf{Inductive Hypothesis} $(n = k)$:

    For an arbitrary natural number $n = k$, we assume that $\sum _{i = 0}^ka^i = \frac{a^{k+1} - 1}{a - 1}$

    \textbf{Inductive Step} $(n = k + 1)$:

    We have to show that $\sum _{i = 0}^{k + 1}a^i = \frac{a^{k+2} - 1}{a - 1}$

    To show this, let's explore the left hand side of the equations:
    \begin{alignat*}{2}
        \sum _{i = 0}^{k + 1}a^i &= \sum _{i = 0}^{k}a^i + a^{k + 1} \\
        &= \frac{a^{k+1} - 1}{a - 1} + a^{k + 1} &&\text{[inductive hypothesis]}\\
        &= \frac{a^{k+1} - 1 + a^{k + 1} \cdot (a - 1)}{a - 1} \\
        &= \frac{a^{k+1} - 1 + a^{k + 2} - a^{k + 1}}{a - 1} \\
        &= \frac{a^{k+2} - 1}{a - 1} &&\text{[subtraction]}
    \end{alignat*}

    Therefore, it is true that $\forall n \in \mathbb{N}, \forall a \in \mathbb{R}\ 1 + a + a^2 + a^3 + ... + a^n = \frac{a^{n + 1} - 1}{a - 1}.\ \square$

    \newpage

    %   ___                         _     _                     ____  
    %  / _ \   _   _    ___   ___  | |_  (_)   ___    _ __     | ___| 
    % | | | | | | | |  / _ \ / __| | __| | |  / _ \  | '_ \    |___ \ 
    % | |_| | | |_| | |  __/ \__ \ | |_  | | | (_) | | | | |    ___) |
    %  \__\_\  \__,_|  \___| |___/  \__| |_|  \___/  |_| |_|   |____/ 
                                                                   

    \question[5]{Prove that $1^3 + 2^3 + 3^3 + ... + n^3 = (\frac{n(n+1)}{2})^2$}

    Proof:

    We proceed by induction on the variable $n$.

    Let $P(n)$ hold the property of the statement for $n$.

    \textbf{Base Case} $(n = 1)$:

    $P(1)$ asserts that $1^3 = (\frac{1(1+1)}{2})^2$

    Taking the right hand side: $(\frac{1(1+1)}{2})^2  = (\frac{1 \cdot 2}{2})^2 = (\frac{2}{2})^2 = 1^2 = 1$

    The base case is proved.

    \textbf{Inductive Hypothesis} $(n = k)$:

    For an arbitrary natural number $n = k$, we assume that $\sum_{i = 1}^k i^3 = (\frac{k(k+1)}{2})^2$

    \textbf{Inductive Step} $(n = k + 1)$:

    We have to show that $\sum_{i = 1}^{k + 1} i^3 = (\frac{(k + 1)(k + 2)}{2})^2$

    To show this, let's explore the left hand side of the equation:
    \begin{alignat*}{2}
        \sum_{i = 1}^{k + 1} i^3 &= \sum_{i = 1}^{k} i^3 + (k + 1)^3 &&\text{[by making next-to-last term explicit]}\\
        &= \left(\frac{k(k+1)}{2}\right)^2 + (k + 1)^3 &&\text{[by inductive hypothesis]}\\
        &= \frac{k^2(k+1)^2}{4} + (k + 1)^3\\
        &= \frac{k^2(k+1)^2+ 4(k + 1)^3}{4}\\
        &= \frac{k^2(k+1)^2+ (k + 1)^2 \cdot 4(k + 1)}{4}\\
        &= \frac{(k + 1)^2 \cdot (k^2 + 4(k + 1))}{4} &&\text{[factoring]}\\
        &= \frac{(k + 1)^2 \cdot (k^2 + 4k + 4)}{4}\\
        &= \frac{(k + 1)^2 \cdot (k + 2)^2}{4}\\
        &= \left(\frac{(k + 1)(k+2)}{2}\right) ^2
    \end{alignat*}

    Therefore, it is true that $1^3 + 2^3 + 3^3 + ... + n^3 = (\frac{n(n+1)}{2})^2$. $\square$

    \newpage

    %   ___                         _     _                      __           
    %  / _ \   _   _    ___   ___  | |_  (_)   ___    _ __      / /_     __ _ 
    % | | | | | | | |  / _ \ / __| | __| | |  / _ \  | '_ \    | '_ \   / _` |
    % | |_| | | |_| | |  __/ \__ \ | |_  | | | (_) | | | | |   | (_) | | (_| |
    %  \__\_\  \__,_|  \___| |___/  \__| |_|  \___/  |_| |_|    \___/   \__,_|
                                                                           

    \question[6]{}

    a) $F_1 + F_2 + F_3 + ... + F_n = F_{n + 2} - 1$

    Proof: 
    
    We proceed by induction on $n$.

    Let $P(n) = \sum_{i = 0}^{n} F_i$, where $F_n = F_{n - 1} + F_{n - 2}$ for all $n > 1$ and $F_0 = 0, F_1 = 1$.

    Defined recursively, $P(n) = P(n - 1) + F_n$.

    We must prove that $P(n) = F_{n + 2} - 1$

    \textbf{Base Case} $(n = 0, 1, 2)$:

    For $n = 0$, we know that $P(0) = 0$, and $F_2 - 1 = (1 + 0) - 1 = 0$. Therefore, $P(0)$ is proved

    For $n = 1$, we know that $P(1) = 1$, and $F_3 - 1 = (1 + 1) - 1 = 1$. Therefore, $P(1)$ is proved

    For $n = 2$, we calculate that $P(2) = 2$, and $F_4 - 1 = (2 + 1) - 1 = 2$. Therefore, $P(2)$ is proved

    \textbf{Inductive Hypothesis} $(n = k)$:

    Let $k$ be any arbitrary natural number and $k > 1$. We assume that $P(k) = F_{k+2} - 1$ for all
    
    natural numbers $i$ from 1 through $k$.

    \textbf{Inductive Step} $(n = k+1)$:

    We need to show that $\sum_{i = 0}^{k+1} F_i = F_{k + 3} - 1$

    Taking the left side:
    \begin{alignat*}{2}
        \sum_{i = 0}^{k+1} F_i &= \sum_{i = 0}^{k} F_i + F_{k + 1}\\
        &= F_{k + 2} - 1 + F_{k + 1}\ &&\text{[inductive hypothesis]}\\
        &= (F_{k + 2} + F_{k + 1}) - 1 \\
        &= F_{k + 3} - 1 &&\text{[from the definition of $F_n$]}
    \end{alignat*}
    
    Therefore, it is true that $F_1 + F_2 + F_3 + ... + F_n = F_{n + 2} - 1$. $\square$

    \newpage

    %   ___                         _     _                      __     _     
    %  / _ \   _   _    ___   ___  | |_  (_)   ___    _ __      / /_   | |__  
    % | | | | | | | |  / _ \ / __| | __| | |  / _ \  | '_ \    | '_ \  | '_ \ 
    % | |_| | | |_| | |  __/ \__ \ | |_  | | | (_) | | | | |   | (_) | | |_) |
    %  \__\_\  \__,_|  \___| |___/  \__| |_|  \___/  |_| |_|    \___/  |_.__/ 
                                                                           

    b) $F_2 + F_4 + F_6 + ... + F_{2n} = F_{2n + 1} - 1$

    Proof:

    We proceed by induction on $n$.

    Let $P(n) = \sum_{i = 0}^{n} F_{2i}$, where $F_n = F_{n - 1} + F_{n - 2}$ for all $n > 1$ and $F_0 = 0, F_1 = 1$.

    Defined recursively, $P(n) = P(n - 1) + F_{2n}$.

    We must prove that $P(n) = F_{2n + 1} - 1$

    \textbf{Base Case} $(n = 1)$:

    For $n = 1$, we calculate that $P(1) = 1$, and $F_3 - 1 = (1 + 1) - 1 = 1$. Therefore, $P(1)$ is proven.

    \textbf{Inductive Hypothesis} $(n = k)$:

    Let $k$ be any arbitrary natural number and $k > 1$. We assume that $P(k) = F_{2k+1} - 1$ for all
    
    natural numbers $i$ from 1 through $k$.

    \textbf{Inductive Step} $(n = k + 1)$:

    We need to show that $\sum_{i = 0}^{k + 1} F_{2i} = F_{2k + 3} - 1$
    
    Taking the left side:
    \begin{alignat*}{2}
        \sum_{i = 0}^{k + 1} F_{2i} &= \sum_{i = 0}^{k} F_{2i} + F_{2k + 2}\\
        &= F_{2k+1} - 1 + F_{2k + 2}\ &&\text{[inductive hypothesis]}\\ 
        &= (F_{2k + 2} + F_{2k+1}) - 1\\ 
        &= F_{2k + 3} - 1\ &&\text{[from the definition of $F_n$]}
    \end{alignat*}

    Therefore, it is true that $F_2 + F_4 + F_6 + ... + F_{2n} = F_{2n + 1} - 1$. $\square$

    \newpage

    %   ___                         _     _                     _____ 
    %  / _ \   _   _    ___   ___  | |_  (_)   ___    _ __     |___  |
    % | | | | | | | |  / _ \ / __| | __| | |  / _ \  | '_ \       / / 
    % | |_| | | |_| | |  __/ \__ \ | |_  | | | (_) | | | | |     / /  
    %  \__\_\  \__,_|  \___| |___/  \__| |_|  \___/  |_| |_|    /_/   
                                                                   

    \question[7]{$a_n = 3 \cdot 2^{n - 1} + 2(-1)^n$ for all $n \in \mathbb{N}$ where $a_n = a_{n - 1} + 2a_{n-2}, a_1 = 1, a_2 = 8, n \geq 3$}

    Proof:

    We proceed by induction on $n$

    Let $P(n)$ hold the property of the statement for $n$.

    \textbf{Base Case} $(n = 1, 2, 3)$

    For $n = 1$, we know that $a_1 = 1$, and $P(1) = 3 \cdot 2^0 + 2(-1)^1 = 1$. Hence, $P(1)$ is proven

    For $n = 2$, we know that $a_2 = 8$, and $P(2) = 3 \cdot 2^1 + 2(-1)^2 = 8$. Hence, $P(2)$ is proven

    For $n = 3$, we calculate $a_3 = 8 + 2(1) = 10$, and $P(3) = 3 \cdot 2^2 + 2(-1)^3 = 10$. $P(3)$ is proven

    \textbf{Inductive Hypothesis} $(n = k)$:

    Let $k$ be an arbitrary natural number greater than 3. We assume that $P(k) = 3 \cdot 2^{n - 1} + 2(-1)^n$

    for all natural numbers $i$ from $1$ to $k$.

    \textbf{Inductive Step} $(n = k + 1)$:

    From the recursive definition of the function, we get that $a_{k + 1} = a_{k} + 2a_{k - 1}$

    We need to show that $a_{k} + 2a_{k - 1} = 3 \cdot 2^{k} + 2(-1)^{k + 1}$

    Taking the left side:
    \begin{alignat*}{2}
        a_{k} + 2a_{k - 1} &= 3 \cdot 2^{k - 1} + 2(-1)^k + 2(3 \cdot 2^{k - 2} + 2(-1)^{k - 1})\ &&\text{[inductive hypothesis]}\\
        &= 3 \cdot 2^{k - 1} + 2(-1)^k + 6 \cdot 2^{k - 2} + 4(-1)^{k - 1}\\
        &= 3 \cdot 2^{k - 1} + 2(-1)^{k} + 3 \cdot 2^{k - 1} - 4(-1)^{k}\\
        &= 6 \cdot 2^{k - 1} - 2(-1)^{k} &&\text{[addition and subtraction]}\\
        &= 3 \cdot 2^{k} + 2(-1)^{k + 1}
    \end{alignat*}

    Therefore, it is true that $a_n = 3 \cdot 2^{n - 1} + 2(-1)^n$ for all $n \in \mathbb{N}$ where $a_n = a_{n - 1} + 2a_{n-2},$
    
    $a_1 = 1, a_2 = 8, n \geq 3$. $\square$

    \newpage

    %   ___                         _     _                      ___  
    %  / _ \   _   _    ___   ___  | |_  (_)   ___    _ __      ( _ ) 
    % | | | | | | | |  / _ \ / __| | __| | |  / _ \  | '_ \     / _ \ 
    % | |_| | | |_| | |  __/ \__ \ | |_  | | | (_) | | | | |   | (_) |
    %  \__\_\  \__,_|  \___| |___/  \__| |_|  \___/  |_| |_|    \___/ 
                                                                   

    \question[8]{Use strong induction to show that every positive integer $n$ can be written as a sum
    of distinct powers of two, that is, as a sum of a subset of the integers $2^0 = 1, 2^1 = 2, 2^2 = 4, ...$}

    Proof:

    We proceed by strong induction on $n$.

    Let $P(n)$ be the proposition that $n$ can be written as a sum of distinct powers of two.

    We will prove $P(n)$ for all $n \in \mathbb{Z}^+$.

    \textbf{Base Case} $(n = 1)$:

    $P(1)$ holds the property of the statement because $2^0 = 1$.

    The base case is proved.

    \textbf{Inductive Hypothesis} $(n = k)$:

    Suppose $k$ is an arbitrary integer greater than 0. Assume that $P(i)$ is true for all $1 \leq i \leq k$ 
    
    for some integer $k$.

    \textbf{Inductive Step} $(n = k + 1)$:

    We have to show that $P(k + 1)$ is even or odd.

    \textit{Case 1:} $k + 1$ is even.

    \tabto{1cm} Since it is even, we can say that $\frac{k + 1}{2}$ is an integer. 

    \tabto{1cm} We know that $1 \leq \frac{k + 1}{2} \leq k$ 

    \tabto{1cm} Based on this, from the inductive hypothesis, we can say that $P(\frac{k + 1}{2})$ is true.

    \tabto{1cm} So this means that $k + 1$ can be written as $2 \cdot \frac{k + 1}{2}$.

    \tabto{1cm} Since we know that $P(\frac{k + 1}{2})$ can be written as a sum of distinct powers of two, we can say 
    
    \tabto{1cm} that all the powers of two in $2 \cdot \frac{k + 1}{2}$ are distinct too.

    \textit{Case 2:} $k + 1$ is odd.
    
    \tabto{1cm} When $k + 1$ is odd, we know that $k$ is even.

    \tabto{1cm} From the inductive hypothesis, we know that $P(k)$ holds.

    \tabto{1cm} Since $k$ is even, the sum cannot include 1, or $2^0$.

    \tabto{1cm} Thus, $k + 1$ can be written as $P(k + 1) = 2^0 + P(k)$.

    \tabto{1cm} So we can say that $P(k + 1)$ can be written as a sum of distinct powers of two.

    Therefore, $P(k + 1)$ is true for all positive integers. $\square$

    \newpage

    %   ___                         _     _                      ___  
    %  / _ \   _   _    ___   ___  | |_  (_)   ___    _ __      / _ \ 
    % | | | | | | | |  / _ \ / __| | __| | |  / _ \  | '_ \    | (_) |
    % | |_| | | |_| | |  __/ \__ \ | |_  | | | (_) | | | | |    \__, |
    %  \__\_\  \__,_|  \___| |___/  \__| |_|  \___/  |_| |_|      /_/ 
                                                                   

    \question[9]{Assume we know that for each natural number $n > 1$, there is a prime number $p$ 
    such that $n < p < 2n$. We call such a prime number a pseudo-prime number. Prove that every 
    natural number $n > 2$ can be written as the summation of distinct pseudo-prime numbers.}

    Proof:

    We proceed by strong induction on $n$.

    Let $P(n)$ be the proposition that $n$ can be written as a sum of distinct pseudo-prime numbers.

    Let $Q(n) = t$ such that $n < t < 2n$ where $t \in \mathbb{N}, t \neq 1, t \neq n,$ and $t \nmid n$ 

    We will prove $P(n)$ for all $n \in \mathbb{N}, n \geq 3$.

    \textbf{Base Case} $(n = 3)$:

    We know that $Q(2) = 3$.

    This means that $P(3)$ holds because $3 = Q(2)$, which means 3 can be written as a sum of distinct 
    
    pseudo-prime numbers. The base case is proved.

    \textbf{Inductive Hypothesis} $(n = k)$:

    For any integer $k \geq 3$ assume that all natural numbers from 3 through $k$ can be written as a 
    
    sum of distinct prime numbers. 

    This means, we assume that any integer $k \geq 3$, $P(i)$ is true for all natural numbers $3 \leq i \leq k$.

    \textbf{Inductive Step} $(n = k + 1)$:

    We need to show that $P(k + 1)$ is true. That means $k + 1$ can be written as a sum of distinct 
    
    pseudo-prime numbers.

    There are two cases: $k + 1$ is a prime itself, or $k + 1$ is a composite number.

    \textit{Case 1:} $k + 1$ is a prime number.

    \tabto{1cm} When $k + 1$ is a prime number, then $P(k + 1)$ will be true.

    \textit{Case 2:} $k + 1$ is a composite number.

    \tabto{1cm} We can say that $Q(\lfloor \frac{k + 1}{2} \rfloor)$ is less than $k + 1$

    \tabto{1cm} So, $k + 1 - Q(\lfloor \frac{k + 1}{2} \rfloor) < k$

    \vspace*{0.1cm}
    
    \tabto{1cm} From the inductive hypothesis we know that $P(k + 1 - Q(\lfloor \frac{k + 1}{2} \rfloor))$ holds true
    \vspace*{0.1cm}

    \tabto{1cm} This means that $P(k + 1) = P(k + 1 - Q(\lfloor \frac{k + 1}{2} \rfloor)) + Q(\lfloor \frac{k + 1}{2} \rfloor)$
    \vspace*{0.1cm}

    \tabto{1cm} Since $Q(\lfloor \frac{k + 1}{2} \rfloor)$ cannot exist in $P(k + 1 - Q(\lfloor \frac{k + 1}{2} \rfloor))$, we know that all primes are distinct.
    \vspace*{0.1cm}

    \tabto{1cm} So, when $k + 1$ is a composite number, $P(k + 1)$ will be true.
    
    Therefore, we know that $k + 1$ can be written as a sum of distinct pseudo-prime numbers. $\square$

\end{document}