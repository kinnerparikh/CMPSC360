\documentclass{article} % This command is used to set the type of document you are working on such as an article, book, or presenation

\usepackage{geometry} % This package allows the editing of the page layout
\usepackage{amsmath}  % This package allows the use of a large range of mathematical formula, commands, and symbols
\usepackage{graphicx}  % This package allows the importing of images
\usepackage{soul}
\usepackage{amsfonts}
\usepackage{dirtytalk}
\usepackage{tabto}
\usepackage{xcolor,colortbl, amssymb}

% https://www.messletters.com/en/big-text/

\newcommand{\question}[2][]{\begin{flushleft}
        \textbf{Question #1}: \text{#2}

\end{flushleft}}

\definecolor{Green}{rgb}{0, 1, 0}
\definecolor{Pink}{rgb}{1, .753, .796}

\newcommand{\sol}{\textbf{Solution}:} %Use if you want a boldface solution line
%\newcommand\tab[1][0.4cm]{\hspace*{#1}}
\newcommand{\maketitletwo}[2][]{\begin{center}
        \Large{\textbf{Homework #1}
            
            CMPSC 360} % Name of course here
        \vspace{5pt}
        
        \normalsize{Kinner Parikh  % Your name here
        
        \today}        % Change to due date if preferred
        \vspace{15pt}
        
\end{center}}
\begin{document}
    \maketitletwo[8]  % Optional argument is assignment number
    %Keep a blank space between maketitletwo and \question[1]
    
    \question[1]{For all $n \in \mathbb{N}$: $3 \mid 2^{2n - 1}$}

    Proof:

    We proceed by induction on the variable $n$.
    
    Let $P(n)$ hold the property of the statement for $n$.

    \textbf{Base Case} (n = 1):

    We need to prove $3 | 2^{2(1) - 1}$


    \question[2]{Show that $n! > 3^n$ for $n \geq 7$}

    Proof: 

    We proceed by induction on the variable $n$.

    \textbf{Base Case} $(n = 7)$:

    We need to prove $7! > 3^7$

    The left hand side of the equation is 5040 and the right hand side is 2187. Since $5040 > 2187$, 
    
    the base case is proved.

    \textbf{Inductive Hypothesis} $(n = k)$:

    For any arbitrary natural number $n = k$ where $k \geq 7$, we assume that $k! > 3^k$

    \textbf{Inductive Step} $(n = k + 1)$:

    We have to show that $(k + 1)! > 3^{k + 1}$

    To show this, let's explore both sides of the equation

    Expanding both sides we get: $(k + 1) \cdot k! > 3 \cdot 3^k$ 

    From the inductive hypothesis, we know that $k! > 3^k$.

    We also know that $k + 1 > 3$ because of the restriction on $k$ that states $k \geq 7$.

    So, we can conclude that $(k + 1) \cdot k! > 3 \cdot 3^k$, which means $(k + 1)! > 3^{k + 1}$ is true.

    Therefore, $\forall n \in \mathbb{N}, k! > 3^k$. $\square$


    \newpage

    \question[3]{For any positive integer $n$, $5 \mid 6^n - 1$}

    Proof:

    We proceed by induction on the variable $n$.

    Let $P(n)$ hold the property of the statement for $n$.

    \textbf{Base Case} $(n = 1)$:

    $P(1)$ asserts that $5 \mid 6^1 - 1$.

    By the definition of divides, $(6^1 - 1) = 5a$ for some $a \in \mathbb{Z}$

    We get, $5 = 5 \cdot 1 = 5$.

    The base case is proved.

    \textbf{Inductive Hypothesis} $(n = k)$:

    For any arbitrary integer $n = k$ where $k \geq 1$, assume that $P(k)$ is true.

    That means $5 \mid 6^k - 1$

    Using the definition of divides, we get $6^k - 1 = 5q$ where $q \in \mathbb{Z}$

    \textbf{Inductive Step} $(n = k + 1)$:

    We have to show that $P(k + 1)$ is true, which means $5 \mid 6^{k + 1} - 1$.

    Expanding the expression, we get:
    \begin{alignat*}{2}
        6^{k + 1} - 1 &= 6 \cdot 6^k - 1 \\
        &=6 \cdot (6^k - 1) + 5\\
        &=6 \cdot 5q + 5&& \text{[inductive step]}\\
        &=5 \cdot (6q + 1)&& \text{[factoring 5 out]} \\
        &=5t&& \text{for some $t \in \mathbb{Z}$ where $t = 6q + 1$}
    \end{alignat*}

    We have  $6^{k + 1} - 1 = 5t$. By definition of divides we get  $5 \mid 6^{k + 1} - 1$.

    Therefore, it is true that $\forall n \in \mathbb{Z}, 5 \mid 6^n - 1.\ \square$ 

    \newpage

    \question[4]{For any $n \in \mathbb{N}$ and any $a \in \mathbb{R}$, prove that $1 + a + a^2 + a^3 + ... + a^n = \frac{a^{n + 1} - 1}{a - 1}$}

    Proof:

    We proceed by induction on the variable $n$.

    Let $P(n)$ hold the property of the statement for $n$.

    \textbf{Base Case} ($n$ = 1):

    $P(1)$ asserts that $1 + a = \frac{a^{1 + 1} - 1}{a-1}$

    Taking the right hand side:
    \begin{alignat*}{2}
        \frac{a^{1 + 1} - 1}{a-1} &= \frac{a^2 - 1}{a-1}\\
        &= \frac{(a+1)(a-1)}{a-1}\ && [\text{factoring}]\\
        &= a + 1&& [\text{divide}]
    \end{alignat*}

    The base case is proved.

    \textbf{Inductive Hypothesis} $(n = k)$:

    For an arbitrary natural number $n = k$, we assume that $\sum _{i = 0}^ka^i = \frac{a^{k+1} - 1}{a - 1}$

    \textbf{Inductive Step} $(n = k + 1)$:

    We have to show that $\sum _{i = 0}^{k + 1}a^i = \frac{a^{k+2} - 1}{a - 1}$

    To show this, let's explore the left hand side of the equations:
    \begin{alignat*}{2}
        \sum _{i = 0}^{k + 1}a^i &= \sum _{i = 0}^{k}a^i + a^{k + 1} \\
        &= \frac{a^{k+1} - 1}{a - 1} + a^{k + 1} &&\text{[injective hypothesis]}\\
        &= \frac{a^{k+1} - 1 + a^{k + 1} \cdot (a - 1)}{a - 1} \\
        &= \frac{a^{k+1} - 1 + a^{k + 2} - a^{k + 1}}{a - 1} \\
        &= \frac{a^{k+2} - 1}{a - 1} &&\text{[subtraction]}\\
    \end{alignat*}

    Therefore, it is true that $\forall n \in \mathbb{N}, \forall a \in \mathbb{R}\ 1 + a + a^2 + a^3 + ... + a^n = \frac{a^{n + 1} - 1}{a - 1}.\ \square$

    \newpage

    \question[5]{Prove that $1^3 + 2^3 + 3^3 + ... + n^3 = (\frac{n(n+1)}{2})^2$}

    Proof:

    We proceed by induction on the variable $n$.

    Let $P(n)$ hold the property of the statement for $n$.

    \textbf{Base Case} $(n = 1)$:

    $P(1)$ asserts that $1^3 = (\frac{1(1+1)}{2})^2$

    Taking the right hand side: $(\frac{1(1+1)}{2})^2  = (\frac{1 \cdot 2}{2})^2 = (\frac{2}{2})^2 = 1^2 = 1$

    The base case is proved.

    \textbf{Inductive Hypothesis} $(n = k)$:

    For an arbitrary natural number $n = k$, we assume that $\sum_{i = 1}^k i^3 = (\frac{k(k+1)}{2})^2$

    \textbf{Inductive Step} $(n = k + 1)$:

    We have to show that $\sum_{i = 1}^{k + 1} i^3 = (\frac{(k + 1)(k + 2)}{2})^2$

    To show this, let's explore the left hand side of the equation:
    \begin{alignat*}{2}
        \sum_{i = 1}^{k + 1} i^3 &= \sum_{i = 1}^{k} i^3 + (k + 1)^3 &&\text{[by making next-to-last term explicit]}\\
        &= \left(\frac{k(k+1)}{2}\right)^2 + (k + 1)^3 &&\text{[by inductive hypothesis]}\\
        &= \frac{k^2(k+1)^2}{4} + (k + 1)^3\\
        &= \frac{k^2(k+1)^2+ 4(k + 1)^3}{4}\\
        &= \frac{k^2(k+1)^2+ (k + 1)^2 \cdot 4(k + 1)}{4}\\
        &= \frac{(k + 1)^2(k^2+ 4(k + 1))}{4} &&\text{[factoring]}\\
        &= \frac{(k + 1)^2(k^2+ 4k + 4)}{4}\\
        &= \frac{(k + 1)^2(k+2)^2}{4}\\
        &= \left(\frac{(k + 1)(k+2)}{2}\right) ^2\\
    \end{alignat*}

    Therefore, it is true that $1^3 + 2^3 + 3^3 + ... + n^3 = (\frac{n(n+1)}{2})^2$. $\square$

\end{document}