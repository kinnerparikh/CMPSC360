\documentclass{article} % This command is used to set the type of document you are working on such as an article, book, or presenation

\usepackage{geometry} % This package allows the editing of the page layout
\usepackage{amsmath}  % This package allows the use of a large range of mathematical formula, commands, and symbols
\usepackage{graphicx}  % This package allows the importing of images
\usepackage{soul}
\usepackage{amsfonts}
\usepackage{dirtytalk}
\usepackage{tabto}
\usepackage{xcolor,colortbl, amssymb}


\newcommand{\question}[2][]{\begin{flushleft}
        \textbf{Question #1}: \textit{#2}

\end{flushleft}}

\definecolor{Green}{rgb}{0, 1, 0}
\definecolor{Pink}{rgb}{1, .753, .796}

\newcommand{\sol}{\textbf{Solution}:} %Use if you want a boldface solution line
%\newcommand\tab[1][0.4cm]{\hspace*{#1}}
\newcommand{\maketitletwo}[2][]{\begin{center}
        \Large{\textbf{Quiz #1}
            
            CMPSC 360} % Name of course here
        \vspace{5pt}
        
        \normalsize{Kinner Parikh  % Your name here
        
        \today}        % Change to due date if preferred
        \vspace{15pt}
        
\end{center}}
\begin{document}
    \maketitletwo[3]  % Optional argument is assignment number
    %Keep a blank space between maketitletwo and \question[1]
    
    \question[1]{}

    $\forall x (P(x) \land Q(x)) \rightarrow R(x)$

    \question[2]{}

    $\neg \forall x \exists y (P(x, y) \rightarrow Q(x, y)) \equiv \exists x \forall y (\neg P(x, y) \lor Q(x, y))$

    $ \exists x \neg \exists y (P(x, y) \rightarrow Q(x, y))$

    $ \exists x\forall y \neg(P(x, y) \rightarrow Q(x, y))$

    $ \exists x\forall y \neg(\neg P(x, y) \lor Q(x, y))$

    $ \exists x\forall y (\neg \neg P(x, y) \land \neg Q(x, y))$

    $ \exists x\forall y (P(x, y) \land \neg Q(x, y))$

    The statement is false

    \question[3]{}

    Domain: $\mathbb{R}$, $P(x, y)$ is $x^2 = 2y$

    $\exists x \forall y P(x, y)$ is false because there is no case where all of $y$ works

    \question[4]{}

    $p \rightarrow q$
    
    \underline{$p \land q$}

    $\therefore \neg q$



    This is not valid.   $((p \rightarrow q) \land (p \land q)) \rightarrow \neg q$

    When $p$ is true and $q$


    \question[5]{}

    1. $p =$ I eat spicy food today, $q =$ My stomach will give me trouble

    \hspace*{0cm}

    \tabto*{0.98cm}$p \rightarrow q$

    \tabto*{0.98cm}\underline{\phantom{...}$\neg q$ \phantom{d}}

    \tabto*{0.98cm} \phantom{} $\therefore \neg p$

    \tabto*{0.98cm}This statement uses the modus tollens inference rule.

    \tabto*{0.98cm}\hspace*{0cm}

    2. $p =$ Penn State will hold a graduation ceremony this Fall
    
    \tabto*{0.98cm}$q =$ all ceremonies will be canceled due to Covid-19

    \hspace*{0cm}

    \tabto*{0.98cm}\underline{\phantom{sf.} $p$ \phantom{d.;}}

    \tabto*{0.98cm}$\therefore p \lor q$

    \tabto*{0.98cm}This statement uses the additive inference rule.
    
    \tabto*{0.98cm}\hspace*{0cm}

    3. $p =$ I am a front end developer, $q =$ I am good at CSS

    \hspace*{0cm}


    \tabto*{0.98cm}\underline{$p \land q$}

    \tabto*{0.98cm}$\therefore p$

    \tabto*{0.98cm}This statement uses the simplification rule.
    
\end{document}